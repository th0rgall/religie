\documentclass[main.tex]{subfiles}

\begin{document}
\section{Korte open vragen}

\begin{examenvraag}
    \begin{vraag}
        Welke twee belangrijke misvattingen over identiteit bestaan er volgens Paul Verhaeghe? Waarom zijn het misvattingen?
    \end{vraag}

    \begin{antwoord}
        \begin{enumerate}
            \item Een identiteit is biologisch vastgelegd.\\
            Dit is fout omdat identiteit een constructie van onze omgeving is. 
            \item Een identiteit is onveranderlijk, diep in mij verborgen.\\
            Die constructie kan gevoelige wijzigingen ondergaan in functie van die omgeving.  Bovendien zit zij niet diep in ons, maar bestaat zij uit vier
			typische verhoudingen tegenover belangrijke anderen:
			\begin{itemize}
				\item ik t.o.v. de ander van het andere geslacht
				\item ik t.o.v. de ander van de andere generatie
				\item ik t.o.v. de ander-gelijke
				\item ik t.o.v. mezelf
			\end{itemize}
        \end{enumerate}
        Geadopteerd kind door een Vlaams koppel. Het kind zal een Vlaamse verhouding
        hebben en niet de verhouding van zijn afkomst. Het is dus niet bepaald door de genen, maar door de
        omgeving. Het is dus wel daadwerkelijk veranderlijk.
        \begin{citaat}{Paul Verhaeghe over identiteit}
            Wij koesteren twee belangrijke misvattingen over onze identiteit.
            De eerste is typisch hedendaags: identiteit zit hem in de genen, in de hersenen en is dus grotendeels biologisch gedetermineerd.
            De tweede sluit daar bij aan: mijn identiteit is een essentiële en dus grotendeels onveranderlijke kern die ergens diep in mij verborgen ligt, en dat min of meer vanaf mijn geboorte.
            Beide opvattingen zijn fout,
        \end{citaat}
    \end{antwoord}
\end{examenvraag}


\begin{examenvraag}
    \begin{vraag}
        Wat leert adoptie ons volgens Paul Verhaeghe over identiteit?
    \end{vraag}

    \begin{antwoord}
        Adoptie toont ons dat iemands identiteit ook, en zelfs veel, afhangt van de omgeving waarin die persoon opgroeit. Het leert ons dus dat identeit veranderlijk en een constructie van de omgeving is en niet in ons zit.
        \begin{citaat}{Paul Verhaeghe over identiteit}
            Mijn identiteit is een constructie van dergelijke verhoudingen tegenover de ander.
            Het woord constructie impliceert dat ik iemand anders had kunnen worden, mocht het constructieproces anders verlopen zijn.
            Het meest overtuigende bewijs daarvoor is adoptie.
            Een kind geboren uit Indiase ouders maar als baby geadopteerd en opgevoed door Vlaamse ouders, wordt een Vlaamse volwassene.
            Begrijp: zij zal die typisch Vlaamse verhoudingen uitbouwen, niet de Indiase.
            Vervang Vlaamse ouders door Hollandse en je krijgt weer een andere constructie.
            Ook het omgekeerde geldt, en dat is een veel moeilijker gedachtenexperiment.
            Beeld u even in dat u, als Nederlandse baby, geadopteerd werd door een moslimkoppel en opgevoed in Soedan. 
            Uw identiteit zou er helemaal anders uitzien, dat wil zeggen, u zou heel andere verhoudingen aannemen tegenover die belangrijke anderen.
        \end{citaat}
    \end{antwoord}
\end{examenvraag}


\begin{examenvraag}
    \begin{vraag}
        Welke drie evolutionaire karakteristieken heeft de mens volgens Paul Verhaeghe verworven?
    \end{vraag}

    \begin{antwoord}
        \begin{enumerate}
            \item We behoren tot de sociale diersoorten.
                \begin{citaat}{Paul Verhaeghe over identiteit}
                    De belangrijkste bepaling daarvan is dat wij tot de sociale diersoorten behoren, niet tot de solitaire.
                    Aristoteles wist dat al toen hij de mens benoemde als een zoön politicon, een dier dat deel uitmaakt van de polis (stadstaat).
                \end{citaat}
            \item We zijn een hiërarchische diersoort.
                \begin{citaat}{Paul Verhaeghe over identiteit}
                    Een tweede evolutionair bepaald kenmerk sluit daar onmiddellijk bij aan: wij zijn een hiërarchische soort, een groep bestaat nooit uit gelijke individuen, maar bevat altijd een sociale stratificatie.
                \end{citaat}
            \item We beschikken over twee verschillende gedragsclusters: \'e\'en gericht op samenleving en \'e\'en gericht op autonomie. Waarvan er \'e\'en dominant is.
                \begin{citaat}{Paul Verhaeghe over identiteit}
                    Frans de Waal (2009) heeft heel mooi en heel overtuigend aangetoond dat primaten twee verschillende gedragsclusters kunnen vertonen.
                    De ene cluster is gericht op samenwerking en delen, de andere op autonomie en nemen.
                    In de eerste kunnen we de basis vinden voor solidariteit, in de andere voor egoïsme.
                \end{citaat}
        \end{enumerate}

    \end{antwoord}
\end{examenvraag}


\begin{examenvraag}
    \begin{vraag}
        Onze identiteit is volgens Paul Verhaeghe het resultaat van twee processen. Welke? Wat houden ze in?
    \end{vraag}

    \begin{antwoord}
        \begin{citaat}{Paul Verhaeghe over identiteit}
            Om daar duidelijkheid in te krijgen, moet ik ingaan op de twee processen die aan het werk zijn binnen de constructie van onze identiteit, meer bepaald identificatie en separatie.
        \end{citaat}
        \begin{enumerate}
            \item \textbf{Identificatie:}
                Identificatie is het proces waarin we `onszelf' worden door ons te spiegelen aan onze omgeving.
                We zijn een spiegeling van onze omgeving (Normatief). We hebben allemaal
                ongeveer een gemeenschappelijke identiteit, dit komt doordat anderen zeggen wat het bv is
                om een man te zijn. Hierdoor is het sterk deterministisch, we kunnen wel nog onze identiteit
                kiezen, maar alleen maar uit de gegeven verhalen.
                \begin{citaat}{Paul Verhaeghe over identiteit}
                    Identificatie heeft dezelfde etymologische grond als identiteit, met name het Latijnse idem, gelijk.
                    Wij worden 'onszelf', tussen aanhalingstekens, door zoveel mogelijk te gelijken op de spiegel die ons voorgehouden wordt.
                    De moderne wetenschappelijke benaming voor identiteit is dan ook spiegeling of `mirroring'.
                \end{citaat}
            \item \textbf{Separatie:}
                Separatie is het proces waarin we `onszelf' worden door ons af te zetten tegen onze omgeving.
                Onder separatie verstaan we de mogelijkheid om te kiezen voor andere
                invullingen van onze identiteit, gebaseerd op alternatieve verhalen. Dit is geen volstrekte
                determinatie, maar ook geen absolute onafhankelijkheid. Deze relatieve keuzevrijheid
                impliceert echter een mate van verantwoordelijkheid voor de gemaakte keuzes.
                \begin{citaat}{Paul Verhaeghe over identiteit}
                Het separatieproces en het bijbehorend streven naar autonomie zijn even belangrijk voor onze identiteit als de identificatie, omdat we daarmee een eigenheid ontwikkelen door ons af te zetten en actief een keuze te maken.
                \end{citaat}
        \end{enumerate}
    \end{antwoord}
\end{examenvraag}


\begin{examenvraag}
    \begin{vraag}
        Dirk De Wachter pleit voor een beetje ongelukkig zijn.
        Waarom doet hij dat? Wat is er dan mis met gelukkig willen zijn?
        Welke paradox stelt De Wachter vast met betrekking tot de manier waarop mensen omgaan met geluk en ongeluk?
    \end{vraag}

    \begin{antwoord}
        Het obsessief bezig zijn met geluk, zoals onze maatschappij dat doet, is ongezond.
        Hij, als psychiater, merkt op dat mensen niet meer om kunnen gaan met ongelukkig zijn.
        Met gelukkig zijn op zich is niets mis, het is de ziekelijke obsessie die een probleem vormt.
        De paradox is dat geluk onze verdienste is maar ongeluk een ziekte die genezen moet worden (door anderen).
        \begin{citaat}{Dirk de Wachter: Pleidoor voor een beetje ongelukkig zijn.}
            Wij willen zo nodig zo gelukkig zijn, dat is een probleem.
            Dat is niet meer normaal hoe gelukkig wij willen zijn.
        \end{citaat}
        \begin{citaat}{Dirk de Wachter: Pleidoor voor een beetje ongelukkig zijn.}
            Geluk is een verdienste\\
            De illusie van de maakbare mens, dat is de meritocratie.
            We hebben ook merite, we hebben verdiensten aan het geluk.
            En de gelukkigen onder ons, die kloppen zich op de borst, die staan vooraan op de speedboot van de snelle maatschappij.
            Die zijn bruingebrand, omringd door langbenige blondines en zij zeggen: kijk eens, ik heb het zelf gedaan.
            Mijn geluk is mijn verdienste.
        \end{citaat}
        \begin{citaat}{Dirk de Wachter: Pleidoor voor een beetje ongelukkig zijn.}
            De meeste pati\"enten aan wie ik vraag `hoe is het met u?' reageren raar.
            Wat vraagt die man nu, wat een rare vraag.
            Die psychiaters zijn toch rare mensen, `hoe is het met u?'
            Het gaat natuurlijk slecht met mij, anders was ik hier niet.
            Geef mij een pil en vlug.
            Dan kan ik mij goed voelen en terug in de wereld doordenderen alsof er niets aan de hand is.
            De pillenmaatschappij lijkt de prijs te zijn die we betalen om de illusie van het geluk hoog te houden.
            Het lastige van pillen is dat ze werken, tijdelijk, een beetje.
            Ze kunnen inderdaad de illusie soms nog wat hoog houden.
            Ik zie zeer veel succesvolle mensen, niet pati\"enten, die dankzij deze pillen kunnen doen alsof ze niet pati\"enten zijn.
            Het is een merkwaardige paradox.
        \end{citaat}
    \end{antwoord}
\end{examenvraag}


\begin{examenvraag}
    \begin{vraag}
        Wat zegt Dirk De Wachter over identiteit vandaag?
    \end{vraag}

    \begin{antwoord}
        Dirk de Wachter argumenteert dat we op weg zijn naar een maatchappij die gekwetst gehecht is.
        Hij beschrijft hoe onze identiteit ites extern wordt. Hetgeen aan de buitenkant te zien valt.
        Een identiteit wordt iets licht, zonder fundament, dat omvergeblazen kan worden bij de minste verschuiving in de situatie.
        \begin{citaat}{Dirk de wachter: `Het belang van identiteit'}
            Het belang van identiteit\\
            Ik denk dat we dreigen te gaan naar een maatschappij die gekwetst gehecht is.
            Waar hechting een probleem wordt.
            [...]
            Mensen verliezen hun identiteit.
            Wie ben ik, waar kom ik vandaan?
            [...]
            Dat lijkt erg te schuiven.
            Onze identiteit is het beeld, is dat hoe we gezien worden in de beeldcultuur, in de reclamecultuur.
            Onze identiteit is ons brilmontuur.
            Onze identiteit is onze modieuze kledij.
            Onze identiteit is dat wat de buitenkant toont en als die buitenkant schuift, door onze gezondheid, door onze economische minder goede situatie, dan verliezen we heel onze identiteit.
            Gehechte identiteit bestaat uit een soort beworteling, uit een soort vastheid, uit een soort zeker zijn.
            Dat niet zomaar makkelijk omver geblazen wordt als er een beetje ongelukkig zijn opdaagt.
            Dat is altijd weer de kern van de zaak.
        \end{citaat}
    \end{antwoord}
\end{examenvraag}


\begin{examenvraag}
    \begin{vraag}
        Volgens Dirk De Wachter zorgt de hedendaagse samenleving ervoor dat we allemaal verlatingsangst hebben. Hoe hebben we dit aannemelijk proberen te maken?
    \end{vraag}

    \begin{antwoord}
        Relaties worden meer een kwestie van afspreken en onderhandelen.
        Er zijn steeds minder omganspatronen om op terug te vallen.
        Hierdoor worden relaties instabieler.
        Zodra er iets beter is kunnen we gewoon naar de volgende gaan.
        Toenemende verlatingsangst is hiervan een ultiem gevolg.
        \begin{citaat}{Slides: `Wat kenmerkt onze maatschappij (27-29)`}
            Binnen relaties kunnen we steeds minder terugvallen op vastliggende omgangsnormen samenleven wordt meer en meer een kwestie van afspreken en onderhandelen:
            \begin{itemize}
                \item ``Relatief vastliggende en vanzelfsprekende voorstellingen over `een goed huwelijk' of `een goede vriendschap' schragen niet langer de wederzijdse (rol)verwachtingen.
                    De concrete interactie-normen worden veeleer doorheen gedurig wederzijds overleg letterlijk gemaakt of geconstrueerd.''
                \item ``Het stijgende ‘maakkarakter’ van sociale relaties'' = de verschuiving van een bevelshuishouding naar een onderhandelingshuishouding:
                    ``In sexueel [sic] en ook wel in ander maatschappelijk verkeer is voor mensen de bewegingsruimte toegenomen, maar onder de nieuwe beperkende voorwaarde dat betrokkenen hun omgang regelen in onderling overleg en naar wederzijdse toestemming.
                    Daardoor worden deze verhoudingen tussen mensen verscheidener en minder voorspelbaar, minder onderworpen aan regels die het resultaat voorschrijven.''
                \item M.a.w. waarden, normen, gedeelde opvattingen en rolpatronen moet voortdurend overlegd worden.
                \item Gevolg = toenemende verlatingsangst (stelling 1 van DW) $\leftarrow$ relaties zijn meer gebaseerd op wat we hebben en doen, minder op wie we zijn (cf. ``liefde is een werkwoord'', ``investeren in je relatie'')
            \end{itemize}
        \end{citaat}
    \end{antwoord}
\end{examenvraag}


\begin{examenvraag}
    \begin{vraag}
        Volgens Dirk De Wachter zorgt de hedendaagse samenleving ervoor dat we allemaal instabiele en intense relaties hebben. Hoe hebben we dit aannemelijk proberen te maken?
    \end{vraag}

    \begin{antwoord}
        Objectieve factoren gaan minder en minder een rol spelen in onze relaties terwijl persoonlijke factoren (meningen, interesses, ...) meer en meer een rol spelen.
        Hierdoor worden relaties intenser.
        Zodra deze subjectieve omstandigheden veranderen kunnen daarom ook de relaties plots instorten.
        Op deze manier worden relaties bovendien instabieler.
        \begin{citaat}{Slides: `Wat kenmerkt onze maatschappij (26)'}
            Grotere sociale (keuze)vrijheid: van “stabiele
            sociabiliteit” naar “vlottende socialiteit”:
            \begin{itemize}
                \item “Bij het aangaan van vriendschaps-, vrijetijds- of liefdesrelaties spelen persoonlijke affiniteiten een almaar grotere rol.
                    ‘Groepsvorming gebeurt in toenemende mate op grond van “subjectieve” factoren,’ … ‘Gedeelde culturele interesses en aspiraties …, eerder dan gegeven “objectieve” factoren als werk of buurt, breng[en] mensen tegenwoordig samen’…”
                \item M.a.w.: onze relaties worden intenser (cf. stelling 2 van DW)\\
                \item Gevolg: onze relaties worden instabieler (cf. stelling 2 van DW):\\
                    “Wanneer het ‘geïndividualiseerde individu’ zich niet langer kan vinden in een liefdes-, huwelijks- of vriendschapsrelatie, gaat z/hij vaak op zoek naar anderen bij wie men zich wel ‘thuis voelt’.
                    Het zich binnen de leefwereld afspelende sociale leven krijgt hierdoor een meer vlottend en minder duurzaam karakter.”
            \end{itemize}
        \end{citaat}
    \end{antwoord}
\end{examenvraag}


\begin{examenvraag}
    \begin{vraag}
        Volgens Dirk De Wachter zorgt de hedendaagse samenleving ervoor dat we ons allemaal afvragen wie we zijn, wat we hier doen en waartoe het allemaal eigenlijk dient. Hoe hebben we dit aannemelijk proberen te maken?
    \end{vraag}

    \begin{antwoord}
        Omdat er geen universeel geldende waarheden en geen funderingen meer zijn, zijn er geen voorgegeven antwoorden meer op de vraag wie we zijn.
        We worden nu alleaal geconfronteerd met de vraag `Wie ben ik', `Wat doe ik hier?', `Waartoe dient het allemaal?'.
        \begin{citaat}{Slides: `Wat kenmerkt onze maatschappij (23)'}
            \begin{itemize}
            \item Gevolgen voor het individu:
            Er zijn geen voorgegeven antwoorden meer op de vraag wie we zijn = “Ver-onzeker-ing” van onze identiteit (OT, p. 66):
            \item Individualisering/detraditionalisering/het einde van de grote verhalen er zijn geen vaste referentiekaders meer en dus geen voorgegeven antwoorden op de vragen wie we zijn en wat onze plaats is in de maatschappij en de wereld
            \item Pluralisering $\rightarrow$ confrontatie met andersheid en met de particulariteit van ons eigen verhaal (onze visie = een bepaald en beperkt perspectief, er zijn andere perspectieven die eigen legitiem blijken te zijn…) we worden allemaal geconfronteerd met de vraag ‘wie ben ik?’, ‘wat doe ik hier?’, ‘waartoe dient het allemaal?’ = stelling 3 van DW
    \end{itemize}
        \end{citaat}
    \end{antwoord}
\end{examenvraag}


\begin{examenvraag}
    \begin{vraag}
        Volgens Dirk De Wachter vertonen we allemaal kenmerken van impulsiviteit. Hoe hebben we dit aannemelijk proberen te maken?
    \end{vraag}

    \begin{antwoord}
        Een toenemende ervaringshonger, de kickcultuur leid ons meteen naar impulsiviteit.
        \begin{citaat}{Slides: `Wat kenmerkt onze maatschappij? (34,35)'}
            Toenemende ervaringshonger:
            \begin{itemize}
                \item = de postmoderne mens wil (en moet!) beleven, ervaren, genieten, plezier maken, voelen dat hij/zij leeft (= de “genotscultuur” of “kickcultuur”)
                \item Kan de postmoderne mens nog verlangen? Of moet het allemaal “nu” en “onmiddellijk”?
                \item De postmoderne mens is altijd op zoek naar het nieuwe, datgene dat hij/zij nog niet meegemaakt heeft (en dat moet steeds straffer en specialer zijn)
                \item De postmoderne mens verdraagt het gewone, het banale, het alledaagse maar moeilijk
                \item DW besluit: We vertonen allemaal kenmerken van impulsiviteit (= stelling 4)
            \end{itemize}
        \end{citaat}
    \end{antwoord}
\end{examenvraag}


\begin{examenvraag}
    \begin{vraag}
        Volgens Dirk De Wachter is zelfdestructie een eigenschap van onze samenleving. Hoe hebben we dit aannemelijke proberen te maken?
    \end{vraag}

    \begin{antwoord}
        Onze toenemende lichaamscultuur brengt ons dicht bij de grens van de zelfmutilatie.
        \begin{citaat}{Slides: `Wat kenmerkt onze maatschappij? (36)'}
            Toenemende lichaamscultuur:
            \begin{itemize}
                \item = het lichaam wordt getraind, versierd, veranderd om erbij te horen \& uit te drukken wie men is/wil zijn (fitness en bodybuilding, tatoeages en piercings, plastische chirurgie en esthetische operaties)
                \item Visie van De Wachter: onze doorgedreven lichaamscultuur brengt ons dicht bij of zelfs over de grens van de automutilatie $\rightarrow$ stelling 5: zelfdestructie is een kenmerk van onze samenleving
            \end{itemize}
        \end{citaat}
    \end{antwoord}
\end{examenvraag}


\begin{examenvraag}
    \begin{vraag}
        Volgens Dirk De Wachter zijn we allemaal affectlabiel. Hoe hebben we dit aannemelijk proberen te maken?
    \end{vraag}

    \begin{antwoord}
        Identiteit is een oppervlaktestructuur geworden, zodra de omgeving veranderd, dreigt de identiteit in te storten.
        We zijn daardoor allemaal affectlabiel.
        \begin{citaat}{Slides: `Wat kenmerkt onze maatschappij? (32)'}
            Gevolg: onze identiteit drijft zeer veranderlijk (flu\"ide, vloeiend, instabiel) te worden:
            [...]
            $\rightarrow$ stelling 6 van DW: we zijn allemaal affectlabiel.
        \end{citaat}
    \end{antwoord}
\end{examenvraag}


\begin{examenvraag}
    \begin{vraag}
        Volgens Dirk De Wachter hebben we allemaal last van zinloosheid en leegte. Hoe hebben we dit aannemelijk proberen te maken?
    \end{vraag}

    \begin{antwoord}
        \'e\'en van de gevolgen van de pluralisering is ori\"entatieverlies.
        Zingeving wordt daardoor steeds meer een probleem.
        \begin{citaat}{Slides: `Wat kenmerkt onze maatschappij? (24)'}
            Ori\"entatieverlied = het wordt moeilijker om zich te situeren in een groter en overkoepelend geheel $\rightarrow$ zingeving/-vinding wordt steeds meer een probleem = stelling 7 van DV: We hebben allemaal last van zinloosheid en leegte.
        \end{citaat}
    \end{antwoord}
\end{examenvraag}


\begin{examenvraag}
    \begin{vraag}
        Volgens Dirk De Wachter leven we in een tijd die doordrenkt is van onaangepaste gressieregulatie. Hoe hebben we dit aannemelijk proberen te maken?
    \end{vraag}

    \begin{antwoord}
        Vanuit de mediacultuur komt er een geweld op als gevolg van zinloosheid en machteloosheid.
        Het geweld is dan een verzet tegen machteloosheid.
        \begin{citaat}{Slides: `Wat kenmerkt onze maatschappij? (37)'}
            En wat met stelling 8, ``We leven in een tijd die doordrenkt is van onaangepaste aggressieregulatie''?
            \begin{itemize}
                \item Hebben we daar wel last van?
                \item Tegenvoorbeelden
                \item Wat is er veranderd?
                    \begin{enumerate}
                        \item Televisie- en computercultuur als voedingsbodem
                        \item Van kwattenkwaad tot hooliganisme als extreme uitingsvorm zinloosheid
                        \item Machteloosheid als overheersend gevoel \& geweld als paradoxaal antwoord
                    \end{enumerate}
            \end{itemize}
        \end{citaat}
    \end{antwoord}
\end{examenvraag}


\begin{examenvraag}
    \begin{vraag}
        Volgens Dirk De Wachter vertonen we allemaal voorbijgaande, stressgebonden paranoïde/dissociatiesymptomen. Hoe hebben we dit aannemelijk proberen te maken?
    \end{vraag}

    \begin{antwoord}
        De hypecultuur zorgt ervoor dat onze emotionele band met gebeurtenissen even vluchtig wordt als de hypes die ze aan de kaart brengen.
        \begin{citaat}{Borderline Times (p. 183)}
            Sommige spontane emotionele reacties lijken gedissocieerde gypes: bepaalde gebeurtenissen worden enorm uitvergroot, en even snel als ze gekomen zijn, lijken ze ook weer verdwenen.
        \end{citaat}
    \end{antwoord}
\end{examenvraag}


\begin{examenvraag}
    \begin{vraag}
        Wat kenmerkt volgens Paul Verhaeghe het neoliberalisme? Wat doet het 
        neoliberalisme met het egoïsme? Welk moreel onderscheid maakt het 
        neoliberalisme?
    \end{vraag}

    \begin{antwoord}
    Het geeft een herdefiniëring van het ego\"isme. Dat gebeurt in 3 stappen:
    \begin{enumerate}
    	\item Het is een typisch menselijk kenmerk
    	\item Het is de hoogste vorm van rationaliteit
    	\item Egoïsme wordt een menselijke deugd
    \end{enumerate}
    
    \begin{citaat}{Paul Verhaeghe over het neoliberalisme}
Binnen het neoliberalisme krijgen we dan ook een doelbewuste herdefiniëring
van het egoïsme. Dit gebeurde in drie stappen. Vooreerst wordt egoïsme voorgesteld als een typisch en
dus normaal menselijk kenmerk, met uitspraken zoals: 'Iedere mens is in wezen een egoïst, wie iets
anders beweert, die liegt' en 'Altruïsme is niets anders dan uitgesteld egoïsme.' Ongeveer iedereen
voelt zich daardoor aangesproken, omdat ongeveer iedereen inderdaad al egoïstisch gehandeld heeft,
en zich daarover schuldig voelt. De tweede stap is dat egoïsme naar voor geschoven wordt als hoogste
vorm van rationaliteit: 'Een rationeel wezen denkt in eerste instantie aan zichzelf, dat is de beste
strategie.' Oef! We hoeven ons toch niet zo schuldig te voelen. Ten slotte wordt egoïsme herdoopt tot
een menselijke deugd, want het precies dit egoïsme dat de vrije markt beter doet functioneren, en dat is
uiteindelijk in het belang van iedereen.
    \end{citaat}
    Het moreel onderscheid tussen de moreel superieure mens (entrepeneur) en de immorele mens (degene die profiteert). Winaars vs. verliezers.
        \begin{citaat}{Paul Verhaeghe over het neoliberalisme}
Op grond daarvan kunnen we de ideale identiteit definiëren: de 'entrepreneur', als de moreel
superieure mens waarop de maatschappij berust. Hardwerkend en toch flexibel, rationeel en dus
instrumenteel-egoïstisch, efficiënt en dus uit op meer winst. Daartegenover staat de immorele mens:
'Wie niet produceert, die profiteert' (Decreus, 2013), met armoede als een symptoom van luiheid. Een
neoliberale maatschappij evolueert vrij snel naar een dergelijke tweedeling: de winnaars bovenaan, die
hun succes enkel te danken hebben aan hun eigen inzet en talent, de verliezers onderaan, wiens verlies
hun eigen schuld is. Op de koop profiteren die verliezers van wat de hardwerkende entrepreneurs
opgebouwd hebben, waardoor ze hun immorele aard nog meer bevestigen. Daarmee ontstaat er meteen
een nieuw schuldgevoel – ik ben een loser, en heb dat alleen maar aan mezelf te danken.
    \end{citaat}
    \end{antwoord}
\end{examenvraag}


\begin{examenvraag}
    \begin{vraag}
        Wat wil Paul Verhaeghe duidelijk maken met de metafoor van het “call 
        center”?
    \end{vraag}
	
    \begin{antwoord}
    Het aatal opties waartussen we door het neoliberale bestel kunnen kiezen is 
    erg beperkt en wordt ons letterlijk gedicteerd.

    
    
    \begin{citaat}{Paul Verhaeghe over het neoliberalisme}
    Dit kadert bij het belangrijkste en vermoedelijk ook het gevaarlijkste 
    neoliberale idee: dat de huidige invulling van de vrije markt ons
	maximale individuele vrijheid zou opleveren en een minimum aan dwang. De facto 
	is het omgekeerde meer en meer waar: de zogenaamde 'vrije' markt dwingt ons in 
	een keurslijf met heel beperkte
	opties. Ik maak graag de vergelijking met een call center, waar een 
	zoetgevooisde stem ons haar
	keuzepalet oplegt: 'Voor optie x, kies 1, voor optie y, kies 2, voor optie z, 		kies drie'. Ik beschouw dit
	als een metafoor voor het neoliberale bestel: het aantal opties waartussen we 
	kunnen kiezen, is niet
	alleen erg beperkt, bovendien worden ze letterlijk gedicteerd.
	\end{citaat}
	
	Ook is er geen centrale autoriteit meer die aanspreekbaar is en ter 	
    verantwoording kan geroepen worden; dit zorgt voor een gevoel van 	
    machteloosheid.	
	
	\begin{citaat}{Paul Verhaeghe over het neoliberalisme}
Het call center als metafoor laat mij toe nog een ander kenmerk naar voor te schuiven: de
anonimiteit, het verdwijnen van een centrale autoriteit die aanspreekbaar is en ter verantwoording
geroepen kan worden als het verkeerd loopt. De verdwijning van een autoriteitsfiguur draagt sterk bij
tot ons veralgemeend gevoel van machteloosheid. We weten niet meer naar wie we kunnen stappen
met onze klacht, vandaar het succes van politieke goeroes die law and order beloven.
	\end{citaat}
	
	\end{antwoord}
\end{examenvraag}


\begin{examenvraag}
    \begin{vraag}
        Welke drie argumenten biedt Paul Verhaeghe om te onderbouwen dat het neoliberalisme negatieve gevolgen heeft?
    \end{vraag}

    \begin{antwoord}
    \begin{enumerate}
    
    	\item Het doet de loonspanning stijgen, wat zorgt voor een negatieve 
    	evolutie in bijna alle psychosociale gezondheidsindicatoren.
    	
    	\begin{citaat}{Paul Verhaeghe over het neoliberalisme}
    		Het eerste ligt op het ruime maatschappelijke vlak. Een neoliberale 
    		maatschappij doet de zogenaamde loonspanning, het onderscheid tussen 
    		de hoogste en laagste inkomens, sterk stijgen. Er is zeer overtuigend 
    		onderzoek dat dit vrij snel leidt tot een negatieve evolutie van 
    		ongeveer alle psychosociale gezondheidsindicatoren: hoe groter de 
    		sociale ongelijkheid, des te meer mentale stoornissen, 
    		tienerzwangerschappen en kindersterfte, agressie, zowel in de 
    		huiskamer als op straat, criminaliteit, drug- en medicatiegebruik 
    		enzovoort.
		\end{citaat}
		
		\item Toenemend gevoel van onbehagen, het andere individu is een 
		potentiële bedreiging. Het zorgt voor individualisme dat veel eenzaamheid 
		oplevert.
		
		\begin{citaat}{Paul Verhaeghe over het neoliberalisme}
Het tweede argument drukt zich vooral uit op het individuele vlak en geeft een verklaring voor
het toenemend gevoel van onbehagen bij het zogenaamde 'vrije' individu. Het neoliberale systeem gaat
op een systematische manier alle sociale verbanden doorknippen. Het is het individu dat geëvalueerd
wordt, dat al dan niet een bonus krijgt of een individuele trajectbegeleiding, enzovoort. Het andere
individu is daarbij een potentiële bedreiging en steeds een concurrent binnen een veralgemeende Rank
and Yank gemeenschap ('Hoeveel 'likes' heb jij op je facebookpagina? En hoeveel bezoekers op je
blog?'). Een typisch gevolg daarvan is de exponentiële toename van contracten, als uitdrukking van
ons veralgemeend wantrouwen. Het doorgedreven individualisme levert vandaag heel veel
eenzaamheid op, als pijnlijkste symptoom van onze tijd. Even terugkeren naar wat ik daarstraks zei
over sociale diersoorten: een individu dat alleen zit, is ofwel ziek ofwel uitgesloten.
		\end{citaat}
		
		\item Het zorgt voor rampzalige ecologische effecten.
		\begin{citaat}{Paul Verhaeghe over het neoliberalisme}
		Tot slot het derde argument, waar er vandaag veel te weinig aandacht aan besteed wordt, met
name de ronduit rampzalige ecologische effecten, met als laatste voorbeeld de CO-2 uitstoot. Het
wordt ronduit cynisch als we zien hoe het neoliberale model dit denkt op te lossen: om de
broeikasgassen te beperken, heeft men er een handeltje van gemaakt, met bijgevolg letterlijk aandelen
in gebakken lucht.
		\end{citaat}
    \end{enumerate}
    \end{antwoord}
\end{examenvraag}


\begin{examenvraag}
    \begin{vraag}
        Waarom is het volgens Paul Verhaeghe een misvatting te geloven dat meer markt zorgt voor meer democratie?
    \end{vraag}

    \begin{antwoord}
   Meer markt leidt tot minder democratie.‭ ‬Dit omdat de overheid het idee van 
   vermarkting nagenoeg volledig heeft overgenomen.‭ ‬Hierdoor zal er steeds meer 
   machtsongelijkheid ontstaan en het verdwijnen van autonomie wordt nog meer in 
   de hand gewerkt.‭ ‬Zo zal men negatieve gevolgen van vrije markt met nog meer 
   vrije markt willen oplossen.
    
\begin{citaat}{Paul Verhaeghe over het neoliberalisme}
De oorzaak daarvan ligt in het feit dat de overheid het idee van de vermarkting zo
ongeveer volledig overgenomen heeft, met als structureel gevolg de installatie van een toenemende
machtsongelijkheid en het verdwijnen van de autonomie.
\end{citaat}
    \end{antwoord}
\end{examenvraag}


\begin{examenvraag}

    \begin{vraag}
        Waarom zorgt meer markt volgens Paul Verhaeghe voor meer regelgeving en procedures?
    \end{vraag}

    \begin{antwoord}
    Dit komt door de koppeling van overheid aan vrije markt, ze zijn volgens hem 
    synoniemen geworden. De neoliberale versie van de vrije markt heeft het 
    sociaal vertrouwen weggeveegd. We zijn een verzameling individuen die alleen 
    uit zijn oop hun eigen voordeel.
    \end{antwoord}
    
    \begin{citaat}{Paul Verhaeghe over het neoliberalisme}
	Vandaag hebben we veel te veel regelgeving – daar zijn we het wel
	over eens – en we leggen daarvoor de schuld bij de overheid. De oplossing 
	wordt dan minder staat en meer vrije markt. Wat we daarbij niet zien, is dat 	
	de overheid ondertussen zo ongeveer synoniem is van de vrije markt, en dat 
	precies die koppeling aan de basis ligt van de exponentiële toename van
	regels. De neoliberale versie van de vrije markt heeft het sociaal vertrouwen 
	weggeveegd en haar eigen uitgangspunt in toenemende mate gerealiseerd, met 
	name een verzameling individuen die alleen uit zijn op hun eigen voordeel. Het 
	antwoord daarop is een groeiend aantal regels, en vooral, een
	toenemende disciplinering van overheidswege. Dit wordt verkocht – soms
	letterlijk – onder het mom van responsabilisering.
    \end{citaat}
\end{examenvraag}


\begin{examenvraag}
    \begin{vraag}
        Hoe sluit onze bespreking van Frans De Waal (begin van thema 3) aan op de bespreking van het neoliberalisme (einde van thema 2)? Hoe hebben we met andere woorden de overgang tussen beide thema’s gemaakt?
    \end{vraag}

    \begin{antwoord}
    Bij het neoliberalisme wordt egoïsme geherdefiniëerd:‭ ‬het is een normaal menselijk kenmerk,‭ ‬het 
    betreft de hoogste vorm van rationaliteit en het is een menselijke deugd.‭ ‬Ook in het artikel 
    van De Waal wordt de mens en dier oorspronkelijk als rivaliserend gezien.‭ ‬Er wordt gesteld dat 
    we‭ ‬enkel aan ons eigen gewin denken.‭ ‬Dit‭ ‬beeld domineert dan ook ons denken over de mens en 
    sluit aan bij de neoliberalistische gedachte.
    \end{antwoord}
\end{examenvraag}


\begin{examenvraag}
    \begin{vraag}
        Waarom kreeg het derde thema als titel “Kwetsbaar geluk” mee?
    \end{vraag}

    \begin{antwoord}
	    \todo[inline]{TODO}
    \end{antwoord}
\end{examenvraag}


\begin{examenvraag}
    \begin{vraag}
        Wat zijn volgens Frans De Waal de drie lagen van empathie?
    \end{vraag}

    \begin{antwoord}
    De Waal beschouwt empathie als een Russische matroesjka.‭ ‬De binnenste laag,‭ ‬het fundament,‭ 
    ‬bestaat uit PAM‭ (‬Perception-Action mechanism‭) ‬en omvat het vermogen om de emotionele toestand 
    van een ander over te nemen.‭ ‬Als tweede laag is er het vermogen om de gevoelens van een ander 
    in‭ ‬te schatten en te beoordelen.‭ ‬Het belangrijkste voorbeeld hiervan is het geven van troost.‭ 
    ‬Ten derde is er de empathische perspectiefname wat het begrijpen wat andere willen of nodig 
    hebben inhoudt.‭ ‬Hierdoor zijn we in staat gericht te helpen.
    \end{antwoord}
\end{examenvraag}


\begin{examenvraag}
    \begin{vraag}
        Wat is altruïsme en op welke manier is empathie hiervan het evolutionair fundament?
    \end{vraag}

    \begin{antwoord}
    	\TODO{not sure if totally correct, vul zeker nog aan}
    	Altru\"{i}sme is synoniem voor onbaatzuchtigheid.
    	In De Waal zijn artikel stond dat altru\"{i}sme een connectie had met empathie.
		Empathie is een positief element geweest voor te overleven. 
    	Het is ook bewezen dat empathie versterkt wordt tussen organismen die sociaal "closer" zijn, die meer gelijkenissen hebben.
    	Nu zou altru\"{i}sme een soort van mechanisme zijn om het organisme meer te kunnen laten identificeren met de ander.
    	Het stimuleert bonden, waardoor een directe reward er niet in zit, maar de reward eerder uitgesteld terugkomt in de vorm
		van wederzijdse empathie. 
		
		Altru\"{i}sme geeft het gevende individu een emotioneel aandeel van de welgesteldheid van de andere.
    	Zie ook "EMPATHY AS EVOLVED PROXIMATE MECHANISM OF DIRECTED ALTRUISM" in artikel De Waal.
    	
    \end{antwoord}
\end{examenvraag}


\begin{examenvraag}
    \begin{vraag}
        Leg uit: “Het is niet omdat biologen het voortdurend over concurrentie hebben dat ze concurrentie aanbevelen” (Frans De Waal).
    \end{vraag}

    \begin{antwoord}
    \todo[inline]{TODO}
    \end{antwoord}
\end{examenvraag}


\begin{examenvraag}
    \begin{vraag}
        De aanleiding voor het vertellen van de parabel van de Barmhartige Samaritaan is de vraag van een wetgeleerde. Om welke vraag gaat het hier? Waarom is dat een strikvraag? Hoe ontsnapt Jezus aan de valstrik?
    \end{vraag}

    \begin{antwoord}
		"Wie is mijn naaste?"
		Het is een strikvraag omdat "naaste" een categorie opmaakt. 
		En wie een categorie gebruikt, sluit automatisch anderen uit.
		De geleerde weet dus maw dat "naaste" de uitsluiting van andere bevolkinsgroepen impliceert.
		Jezus ontsnapt als een ware actieheld door het begrip uit te leggen adh van het verhaal van de barmhartige samaritaan.
		Hierdoor wordt de vraag naar het object van de naastenliefde ("Wie is mijn naaste?")  vervangen door een vraag naar het subject van de naastenliefde.
		
		    
    
    \end{antwoord}
\end{examenvraag}


\begin{examenvraag}
    \begin{vraag}
        Wat wordt bedoeld met “het lichamelijk fundament van de intermenselijke ethiek”?
    \end{vraag}

    \TODO{pretty sure dat dit geen 7/10 waardig antwoord is}
    \begin{antwoord}
       We worden door het lijden van een ander getroffen omdat we trefbaar zijn, omdat we door en door lichamelijk zijn.
       Denk aan de uitspraak "daar draait mijn maag van om". 
       Je ziet iets waardoor je niet "geestelijk" maar lichamelijk geraakt wordt.
       Ons lichaam is gevoelig voor het lijden van de medemens.
       Net daarom voelen de 3 passanten in het verhaal (Samaritaan) hun eigen verantwoordelijk.
       (Of ze de verantwoodelijkheid opnemen is iets anders.)
       Die verantwoordelijkheid zorgt voor die intermenselijke ethiek. (???)

    \end{antwoord}
\end{examenvraag}


\begin{examenvraag}
    \begin{vraag}
        Wat is de ethische grondervaring?
    \end{vraag}
        Zie p83 barmhartige samaritaan.

    \begin{antwoord}
        Zie p83 barmhartige samaritaan.
        De ethische verhouding met de ander begint niet als positieve beweging, maar eerder als een schokervaring. 
        Iemand zo zien lijden roept dingen in je op.
        Je MAG de man niet laten liggen, maar eigenlijk KAN je het toch.
        Als ik beslis de man te laten liggen, besef ik dat wat dus in feite kan, eigenlijk niet mag. 
        En dat is de kern van de ethische grondervaring.
    \end{antwoord}
\end{examenvraag}


\begin{examenvraag}
    \begin{vraag}
        Waarom is de naastenliefde een gebod?
    \end{vraag}

    \begin{antwoord}
    \todo[inline]{TODO}
    \end{antwoord}
\end{examenvraag}


\begin{examenvraag}
    \begin{vraag}
        Op welke manier maakt het verschijnen van de noodlijdende ander mij uniek?
    \end{vraag}

    \begin{antwoord}
        \TODO{p87 barmhartige samaritaan}
    \end{antwoord}
\end{examenvraag}


\begin{examenvraag}
    \begin{vraag}
        Wat zijn de kenmerken van de barmhartigheid?
    \end{vraag}

    \begin{antwoord}
    \todo[inline]{TODO}
    \end{antwoord}
\end{examenvraag}


\begin{examenvraag}
    \begin{vraag}
        In de hedendaagse bewerking van de parabel van de Barmhartige Samaritaan die tijdens de les getoond werd, wordt ons aangeraden iedereen lief te hebben alsof ze dood zullen zijn tegen middernacht. Hoe kunnen we dit begrijpen?
    \end{vraag}

    \begin{antwoord}
    \todo[inline]{TODO}
    \end{antwoord}
\end{examenvraag}


\begin{examenvraag}
    \begin{vraag}
        Bespreek de grenzen van de barmhartigheid in verwijzing naar Genesis, hoofdstuk 12.
    \end{vraag}

    \begin{antwoord}
        \begin{citaat}{Over de fundamenten van de ethiek(22)}
            Twee (complementaire) grenzen:
            \begin{itemize}
                \item 1. De eigen persoonswaarde en integriteit als grens
                \item 2. Verantwoordelijkheid in de derde persoon
            \end{itemize}
        \end{citaat}

        1. Niemand hoeft afstand te doen van zijn/haar menselijke waardigheid – ik mag er niet
        mee instemmen mezelf louter als middel te behandelen en zo mezelf te vernietigen (cf. Kants categorische imperatief).
        Niemand kan verplicht worden het beeld van god in zich vergeten.
        2. Helemaal ingaan op het appel van \'{e}\'{e}n noodlijdende is onrecht doen aan de derden.


    \end{antwoord}
\end{examenvraag}


\begin{examenvraag}
    \begin{vraag}
        Wat is de band tussen de parabel van de Barmhartige Samaritaan en Schindler’s List?
    \end{vraag}

    \begin{antwoord}
    \todo[inline]{TODO}
    \end{antwoord}
\end{examenvraag}


\begin{examenvraag}
    \begin{vraag}
        Waarom werden tijdens de les fragmenten uit Schindler’s List getoond? Wat hebben we geleerd uit deze fragmenten?
    \end{vraag}

    \begin{antwoord}
    \todo[inline]{TODO}
    \end{antwoord}
\end{examenvraag}


\begin{examenvraag}
    \begin{vraag}
        Wat is de visie van Francis Collins op de verhouding tussen geloof en wetenschap?
    \end{vraag}			
    			
    \begin{antwoord}
    	\TODO{vul aan, kijk na, ween}
    	Hanteert het harmoniemodel: verzoening wetenschap met geloof.
    	Daar waar "harde" wetenschappers doorgaans voor materialisme pleiten, zien we binnen dit model af van die ideologie. 
    	Natuurwetten worden geaccepteerd, maar krijgen een "God-smaakje". De natuurwetten zijn een middel van God om de wereld etc in stand te houden. 
    	Evolutieleer staat haaks op het scheppingsverhaal, echter in dit model kan evolutieleer toch een plaats krijgen binnen het geloof.
    	Dit omdat evolutieleer een middel was van god.
    	Evolutieleer wordt ook beschouwd als iets doelloos. Volkomen random events, die random resultaten opleverden,  and now we're here. 
    	Er is geen randomness, ook hier wordt ruimte gemaakt voor het spirituele. 
    	God dirigeerde de processen.
    	Merk wel op dat in dit model er bijvoorbeeld wordt afgestapt van het feit dat de aarde maar 2000 jaar oud zou zijn. 
    	In dit model durven ze dingen aan te nemen van de wetenschap.
    	De wetenschap en de religie vullen(?) elkaar een beetje aan? i dunno man, such vague, much bs,  wow.
    	
    	Informatie deels te halen uit slides Geloof en Wetenschap(9-10).
    	
    \end{antwoord}
\end{examenvraag}


\begin{examenvraag}
    \begin{vraag}
        Hoe probeert Stephen Jay Gould het conflict tussen geloof en wetenschap op te lossen? Slaagt hij in zijn opzet?
    \end{vraag}

    \begin{antwoord}
        \TODO{Slaagt hij?}
        Door het introduceren van het NOMA-principe. (Non-overlapping magistria) 
        Hij maakt een onderscheid tussen natuurwetenschappen enerzijds en religie, zingeving en ethiek anderzijds.
        Waar men met natuurwetenschappen de realiteit probeert te verklaren, proberen religie, zingeving en ethiek 		   levensvragen te beantwoorden. 
        Het zijn gescheiden talen, met een complementaire functie.
        Het conflict wordt in dit geval opgelost door te stellen dat beide "talen" zo fundamenteel anders zijn, dat ze eenvoudigweg niet met elkaar in conflict kunnen komen. (Kloofmodel)
    \end{antwoord}
\end{examenvraag}


\begin{examenvraag}
    \begin{vraag}
        Wat is de visie van Herman De Dijn op de verhouding tussen geloof en wetenschap? Formuleer een mogelijke kritiek op zijn visie.
    \end{vraag}

    \begin{antwoord}
    	\TODO{kritiek, vul aan}
    	De Dijn pleit, zoals Gould, voor een (onder)scheiding tussen geloof en wetenschap. 
    	Waar wetenschap over de wereld probeert te leren, probeert geloof te leren hoe we in deze wereld moeten leven.
    	
    		\begin{citaat}{slides Geloof en Wetenschap(8)}
    			[1] Religie [heeft] niet primair te maken met het hebben van of met het zoeken naarde wetmatigheden in de werkelijkheid, ook niet met het zoeken naar de waarheidbetreffende het universum of betreffende de oorsprong van alles. Religie betreft in deeerste plaats een manier van leven waarin men bekwaam en bereid is zich ten minsteaf en toe op een bepaalde wijze te confronteren met [2] de beperktheid of eindigheidvan zichzelf, van zijn geliefden en van alles wat we de moeite waard vinden. Dieconfrontatie [...] vergt niet zozeer steeds dieper gravend wetenschappelijk onderzoek,maar de bereidheid tot [3] bepaalde houdingen, tot gepaste manieren van reageren,kortom tot een juiste manier van leven
    		\end{citaat}
    	
    \end{antwoord}
\end{examenvraag}


\begin{examenvraag}
    \begin{vraag}
        Hoe probeert John F. Haught het conflict tussen evolutietheorie en christendom op te lossen? Formuleer een mogelijke kritiek op zijn poging.
    \end{vraag}

    \begin{antwoord}
    \todo[inline]{TODO}
    \end{antwoord}
\end{examenvraag}


\begin{examenvraag}
    \begin{vraag}
    Hoe denkt het differentiemodel over de verhouding tussen geloof en wetenschap?
    \end{vraag}

    \begin{antwoord}
    \todo[inline]{TODO}
    \end{antwoord}
\end{examenvraag}


\end{document}
