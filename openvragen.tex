\documentclass[main.tex]{subfiles}

\begin{document}
\section{Korte open vragen}

\begin{examenvraag}
    \begin{vraag}
        Welke twee belangrijke misvattingen over identiteit bestaan er volgens Paul Verhaeghe? Waarom zijn het misvattingen?
    \end{vraag}

    \begin{antwoord}
    \end{antwoord}
\end{examenvraag}


\begin{examenvraag}
    \begin{vraag}
        Wat leert adoptie ons volgens Paul Verhaeghe over identiteit?
    \end{vraag}

    \begin{antwoord}
    \end{antwoord}
\end{examenvraag}


\begin{examenvraag}
    \begin{vraag}
        Welke drie evolutionaire karakteristieken heeft de mens volgens Paul Verhaeghe verworven?
    \end{vraag}

    \begin{antwoord}
    \end{antwoord}
\end{examenvraag}


\begin{examenvraag}
    \begin{vraag}
        Onze identiteit is volgens Paul Verhaeghe het resultaat van twee processen. Welke? Wat houden ze in?
    \end{vraag}

    \begin{antwoord}
    \end{antwoord}
\end{examenvraag}


\begin{examenvraag}
    \begin{vraag}
        Dirk De Wachter pleit voor een beetje ongelukkig zijn. Waarom doet hij dat? Wat is er dan mis met gelukkig willen zijn? Welke paradox stelt De Wachter vast met betrekking tot de manier waarop mensen omgaan met geluk en ongeluk?
    \end{vraag}

    \begin{antwoord}
    \end{antwoord}
\end{examenvraag}


\begin{examenvraag}
    \begin{vraag}
        Wat zegt Dirk De Wachter over identiteit vandaag?
    \end{vraag}

    \begin{antwoord}
    \end{antwoord}
\end{examenvraag}


\begin{examenvraag}
    \begin{vraag}
        Volgens Dirk De Wachter zorgt de hedendaagse samenleving ervoor dat we allemaal verlatingsangst hebben. Hoe hebben we dit aannemelijk proberen te maken?
    \end{vraag}

    \begin{antwoord}
    \end{antwoord}
\end{examenvraag}


\begin{examenvraag}
    \begin{vraag}
        Volgens Dirk De Wachter zorgt de hedendaagse samenleving ervoor dat we allemaal instabiele en intense relaties hebben. Hoe hebben we dit aannemelijk proberen te maken?
    \end{vraag}

    \begin{antwoord}
    \end{antwoord}
\end{examenvraag}


\begin{examenvraag}
    \begin{vraag}
        Volgens Dirk De Wachter zorgt de hedendaagse samenleving ervoor dat we ons allemaal afvragen wie we zijn, wat we hier doen en waartoe het allemaal eigenlijk dient. Hoe hebben we dit aannemelijk proberen te maken?
    \end{vraag}

    \begin{antwoord}
    \end{antwoord}
\end{examenvraag}


\begin{examenvraag}
    \begin{vraag}
        Volgens Dirk De Wachter vertonen we allemaal kenmerken van impulsiviteit. Hoe hebben we dit aannemelijk proberen te maken?
    \end{vraag}

    \begin{antwoord}
    \end{antwoord}
\end{examenvraag}


\begin{examenvraag}
    \begin{vraag}
        Volgens Dirk De Wachter is zelfdestructie een eigenschap van onze samenleving. Hoe hebben we dit aannemelijke proberen te maken?
    \end{vraag}

    \begin{antwoord}
    \end{antwoord}
\end{examenvraag}


\begin{examenvraag}
    \begin{vraag}
        Volgens Dirk De Wachter zijn we allemaal affectlabiel. Hoe hebben we dit aannemelijk proberen te maken?
    \end{vraag}

    \begin{antwoord}
    \end{antwoord}
\end{examenvraag}


\begin{examenvraag}
    \begin{vraag}
        Volgens Dirk De Wachter hebben we allemaal last van zinloosheid en leegte. Hoe hebben we dit aannemelijk proberen te maken?
    \end{vraag}

    \begin{antwoord}
    \end{antwoord}
\end{examenvraag}


\begin{examenvraag}
    \begin{vraag}
        Volgens Dirk De Wachter leven we in een tijd die doordrenkt is van onaangepaste gressieregulatie. Hoe hebben we dit aannemelijk proberen te maken?
    \end{vraag}

    \begin{antwoord}
    \end{antwoord}
\end{examenvraag}


\begin{examenvraag}
    \begin{vraag}
        Volgens Dirk De Wachter vertonen we allemaal voorbijgaande, stressgebonden paranoïde/dissociatiesymptomen. Hoe hebben we dit aannemelijk proberen te maken?
    \end{vraag}

    \begin{antwoord}
    \end{antwoord}
\end{examenvraag}


\begin{examenvraag}
    \begin{vraag}
        Wat kenmerkt volgens Paul Verhaeghe het neoliberalisme? Wat doet het neoliberalisme met het egoïsme? Welk moreel onderscheid maakt het neoliberalisme?
    \end{vraag}

    \begin{antwoord}
    \end{antwoord}
\end{examenvraag}


\begin{examenvraag}
    \begin{vraag}
        Wat wil Paul Verhaeghe duidelijk maken met de metafoor van het “call center”?
    \end{vraag}

    \begin{antwoord}
    \end{antwoord}
\end{examenvraag}


\begin{examenvraag}
    \begin{vraag}
        Welke drie argumenten biedt Paul Verhaeghe om te onderbouwen dat het neoliberalisme negatieve gevolgen heeft?
    \end{vraag}

    \begin{antwoord}
    \end{antwoord}
\end{examenvraag}


\begin{examenvraag}
    \begin{vraag}
        Waarom is het volgens Paul Verhaeghe een misvatting te geloven dat meer markt zorgt voor meer democratie?
    \end{vraag}

    \begin{antwoord}
    \end{antwoord}
\end{examenvraag}


\begin{examenvraag}
    \begin{vraag}
        Waarom zorgt meer markt volgens Paul Verhaeghe voor meer regelgeving en procedures?
    \end{vraag}

    \begin{antwoord}
    \end{antwoord}
\end{examenvraag}


\begin{examenvraag}
    \begin{vraag}
        Hoe sluit onze bespreking van Frans De Waal (begin van thema 3) aan op de bespreking van het neoliberalisme (einde van thema 2)? Hoe hebben we met andere woorden de overgang tussen beide thema’s gemaakt?
    \end{vraag}

    \begin{antwoord}
    \end{antwoord}
\end{examenvraag}


\begin{examenvraag}
    \begin{vraag}
        Waarom kreeg het derde thema als titel “Kwetsbaar geluk” mee?
    \end{vraag}

    \begin{antwoord}
    \end{antwoord}
\end{examenvraag}


\begin{examenvraag}
    \begin{vraag}
        Wat zijn volgens Frans De Waal de drie lagen van empathie?
    \end{vraag}

    \begin{antwoord}
    \end{antwoord}
\end{examenvraag}


\begin{examenvraag}
    \begin{vraag}
        Wat is altruïsme en op welke manier is empathie hiervan het evolutionair fundament?
    \end{vraag}

    \begin{antwoord}
    \end{antwoord}
\end{examenvraag}


\begin{examenvraag}
    \begin{vraag}
        Leg uit: “Het is niet omdat biologen het voortdurend over concurrentie hebben dat ze concurrentie aanbevelen” (Frans De Waal).
    \end{vraag}

    \begin{antwoord}
    \end{antwoord}
\end{examenvraag}


\begin{examenvraag}
    \begin{vraag}
        De aanleiding voor het vertellen van de parabel van de Barmhartige Samaritaan is de vraag van een wetgeleerde. Om welke vraag gaat het hier? Waarom is dat een strikvraag? Hoe ontsnapt Jezus aan de valstrik?
    \end{vraag}

    \begin{antwoord}
    \end{antwoord}
\end{examenvraag}


\begin{examenvraag}
    \begin{vraag}
        Wat wordt bedoeld met “het lichamelijk fundament van de intermenselijke ethiek”?
    \end{vraag}

    \begin{antwoord}
    \end{antwoord}
\end{examenvraag}


\begin{examenvraag}
    \begin{vraag}
        Wat is de ethische grondervaring?
    \end{vraag}

    \begin{antwoord}
    \end{antwoord}
\end{examenvraag}


\begin{examenvraag}
    \begin{vraag}
        Waarom is de naastenliefde een gebod?
    \end{vraag}

    \begin{antwoord}
    \end{antwoord}
\end{examenvraag}


\begin{examenvraag}
    \begin{vraag}
        Op welke manier maakt het verschijnen van de noodlijdende ander mij uniek?
    \end{vraag}

    \begin{antwoord}
    \end{antwoord}
\end{examenvraag}


\begin{examenvraag}
    \begin{vraag}
        Wat zijn de kenmerken van de barmhartigheid?
    \end{vraag}

    \begin{antwoord}
    \end{antwoord}
\end{examenvraag}


\begin{examenvraag}
    \begin{vraag}
        In de hedendaagse bewerking van de parabel van de Barmhartige Samaritaan die tijdens de les getoond werd, wordt ons aangeraden iedereen lief te hebben alsof ze dood zullen zijn tegen middernacht. Hoe kunnen we dit begrijpen?
    \end{vraag}

    \begin{antwoord}
    \end{antwoord}
\end{examenvraag}


\begin{examenvraag}
    \begin{vraag}
        Bespreek de grenzen van de barmhartigheid in verwijzing naar Genesis, hoofdstuk 12.
    \end{vraag}

    \begin{antwoord}
    \end{antwoord}
\end{examenvraag}


\begin{examenvraag}
    \begin{vraag}
        Wat is de band tussen de parabel van de Barmhartige Samaritaan en Schindler’s List?
    \end{vraag}

    \begin{antwoord}
    \end{antwoord}
\end{examenvraag}


\begin{examenvraag}
    \begin{vraag}
        Waarom werden tijdens de les fragmenten uit Schindler’s List getoond? Wat hebben we geleerd uit deze fragmenten?
    \end{vraag}

    \begin{antwoord}
    \end{antwoord}
\end{examenvraag}


\begin{examenvraag}
    \begin{vraag}
        Wat is de visie van Francis Collins op de verhouding tussen geloof en wetenschap?
    \end{vraag}

    \begin{antwoord}
    \end{antwoord}
\end{examenvraag}


\begin{examenvraag}
    \begin{vraag}
        Hoe probeert Stephen Jay Gould het conflict tussen geloof en wetenschap op te lossen? Slaagt hij in zijn opzet?
    \end{vraag}

    \begin{antwoord}
    \end{antwoord}
\end{examenvraag}


\begin{examenvraag}
    \begin{vraag}
        Wat is de visie van Herman De Dijn op de verhouding tussen geloof en wetenschap? Formuleer een mogelijke kritiek op zijn visie.
    \end{vraag}

    \begin{antwoord}
    \end{antwoord}
\end{examenvraag}


\begin{examenvraag}
    \begin{vraag}
        Hoe probeert John F. Haught het conflict tussen evolutietheorie en christendom op te lossen? Formuleer een mogelijke kritiek op zijn poging.
    \end{vraag}

    \begin{antwoord}
    \end{antwoord}
\end{examenvraag}


\begin{examenvraag}
    \begin{vraag}
        Hoe denkt het differentiemodel over de verhouding tussen geloof en wetenschap?
    \end{vraag}

    \begin{antwoord}
    \end{antwoord}
\end{examenvraag}


\end{document}
