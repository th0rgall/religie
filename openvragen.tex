\documentclass[main.tex]{subfiles}

\begin{document}
\section{Korte open vragen}

\begin{examenvraag}
    \begin{vraag}
        Welke twee belangrijke misvattingen over identiteit bestaan er volgens Paul Verhaeghe? Waarom zijn het misvattingen?
    \end{vraag}

    \begin{antwoord}
        \begin{enumerate}
            \item Een identiteit is biologisch vastgelegd.
            \item Een identiteit is onveranderlijk.
        \end{enumerate}
        \begin{citaat}{Paul Verhaeghe over identiteit}
            Wij koesteren twee belangrijke misvattingen over onze identiteit.
            De eerste is typisch hedendaags: identiteit zit hem in de genen, in de hersenen en is dus grotendeels biologisch gedetermineerd.
            De tweede sluit daar bij aan: mijn identiteit is een essentiële en dus grotendeels onveranderlijke kern die ergens diep in mij verborgen ligt, en dat min of meer vanaf mijn geboorte.
            Beide opvattingen zijn fout,
        \end{citaat}
    \end{antwoord}
\end{examenvraag}


\begin{examenvraag}
    \begin{vraag}
        Wat leert adoptie ons volgens Paul Verhaeghe over identiteit?
    \end{vraag}

    \begin{antwoord}
        Het biologisch fundament van identiteit is niet de enige factor in de constructie.
        Adoptie toont ons dat iemands identiteit ook, en zelfs veel, afhangt van de omgeving waarin die persoon opgroeit.
        \begin{citaat}{Paul Verhaeghe over identiteit}
            Mijn identiteit is een constructie van dergelijke verhoudingen tegenover de ander.
            Het woord constructie impliceert dat ik iemand anders had kunnen worden, mocht het constructieproces anders verlopen zijn.
            Het meest overtuigende bewijs daarvoor is adoptie.
            Een kind geboren uit Indiase ouders maar als baby geadopteerd en opgevoed door Vlaamse ouders, wordt een Vlaamse volwassene.
            Begrijp: zij zal die typisch Vlaamse verhoudingen uitbouwen, niet de Indiase.
            Vervang Vlaamse ouders door Hollandse en je krijgt weer een andere constructie.
            Ook het omgekeerde geldt, en dat is een veel moeilijker gedachtenexperiment.
            Beeld u even in dat u, als Nederlandse baby, geadopteerd werd door een moslimkoppel en opgevoed in Soedan. 
            Uw identiteit zou er helemaal anders uitzien, dat wil zeggen, u zou heel andere verhoudingen aannemen tegenover die belangrijke anderen.
        \end{citaat}
    \end{antwoord}
\end{examenvraag}


\begin{examenvraag}
    \begin{vraag}
        Welke drie evolutionaire karakteristieken heeft de mens volgens Paul Verhaeghe verworven?
    \end{vraag}

    \begin{antwoord}
        \begin{enumerate}
            \item We behoren tot de sociale diersoorten.
                \begin{citaat}{Paul Verhaeghe over identiteit}
                    De belangrijkste bepaling daarvan is dat wij tot de sociale diersoorten behoren, niet tot de solitaire.
                    Aristoteles wist dat al toen hij de mens benoemde als een zoön politicon, een dier dat deel uitmaakt van de polis (stadstaat).
                \end{citaat}
            \item We zijn een hiërarchische diersoort.
                \begin{citaat}{Paul Verhaeghe over identiteit}
                    Een tweede evolutionair bepaald kenmerk sluit daar onmiddellijk bij aan: wij zijn een hiërarchische soort, een groep bestaat nooit uit gelijke individuen, maar bevat altijd een sociale stratificatie.
                \end{citaat}
            \item We beschikken over twee verschillende gedragsclusters: \'e\'en gericht op samenleving en \'e\'en gericht op autonomie.
                \begin{citaat}{Paul Verhaeghe over identiteit}
                    Frans de Waal (2009) heeft heel mooi en heel overtuigend aangetoond dat primaten twee verschillende gedragsclusters kunnen vertonen.
                    De ene cluster is gericht op samenwerking en delen, de andere op autonomie en nemen.
                    In de eerste kunnen we de basis vinden voor solidariteit, in de andere voor egoïsme.
                \end{citaat}
        \end{enumerate}

    \end{antwoord}
\end{examenvraag}


\begin{examenvraag}
    \begin{vraag}
        Onze identiteit is volgens Paul Verhaeghe het resultaat van twee processen. Welke? Wat houden ze in?
    \end{vraag}

    \begin{antwoord}
        \begin{citaat}{Paul Verhaeghe over identiteit}
            Om daar duidelijkheid in te krijgen, moet ik ingaan op de twee processen die aan het werk zijn binnen de constructie van onze identiteit, meer bepaald identificatie en separatie.
        \end{citaat}
        \begin{enumerate}
            \item Identificatie.
                Identificatie is het proces waarin we `onszelf' worden door ons te spiegelen aan onze omgeving.
                \begin{citaat}{Paul Verhaeghe over identiteit}
                    Identificatie heeft dezelfde etymologische grond als identiteit, met name het Latijnse idem, gelijk.
                    Wij worden 'onszelf', tussen aanhalingstekens, door zoveel mogelijk te gelijken op de spiegel die ons voorgehouden wordt.
                    De moderne wetenschappelijke benaming voor identiteit is dan ook spiegeling of `mirroring'.
                \end{citaat}
            \item Separatie.
                Separatie is het proces waarin we `onszelf' worden door ons af te zetten tegen onze omgeving.
                \begin{citaat}{Paul Verhaeghe over identiteit}
                Het separatieproces en het bijbehorend streven naar autonomie zijn even belangrijk voor onze identiteit als de identificatie, omdat we daarmee een eigenheid ontwikkelen door ons af te zetten en actief een keuze te maken.
                \end{citaat}
        \end{enumerate}
    \end{antwoord}
\end{examenvraag}


\begin{examenvraag}
    \begin{vraag}
        Dirk De Wachter pleit voor een beetje ongelukkig zijn. Waarom doet hij dat? Wat is er dan mis met gelukkig willen zijn? Welke paradox stelt De Wachter vast met betrekking tot de manier waarop mensen omgaan met geluk en ongeluk?
    \end{vraag}

    \begin{antwoord}
    \end{antwoord}
\end{examenvraag}


\begin{examenvraag}
    \begin{vraag}
        Wat zegt Dirk De Wachter over identiteit vandaag?
    \end{vraag}

    \begin{antwoord}
    \end{antwoord}
\end{examenvraag}


\begin{examenvraag}
    \begin{vraag}
        Volgens Dirk De Wachter zorgt de hedendaagse samenleving ervoor dat we allemaal verlatingsangst hebben. Hoe hebben we dit aannemelijk proberen te maken?
    \end{vraag}

    \begin{antwoord}
    \end{antwoord}
\end{examenvraag}


\begin{examenvraag}
    \begin{vraag}
        Volgens Dirk De Wachter zorgt de hedendaagse samenleving ervoor dat we allemaal instabiele en intense relaties hebben. Hoe hebben we dit aannemelijk proberen te maken?
    \end{vraag}

    \begin{antwoord}
    \end{antwoord}
\end{examenvraag}


\begin{examenvraag}
    \begin{vraag}
        Volgens Dirk De Wachter zorgt de hedendaagse samenleving ervoor dat we ons allemaal afvragen wie we zijn, wat we hier doen en waartoe het allemaal eigenlijk dient. Hoe hebben we dit aannemelijk proberen te maken?
    \end{vraag}

    \begin{antwoord}
    \end{antwoord}
\end{examenvraag}


\begin{examenvraag}
    \begin{vraag}
        Volgens Dirk De Wachter vertonen we allemaal kenmerken van impulsiviteit. Hoe hebben we dit aannemelijk proberen te maken?
    \end{vraag}

    \begin{antwoord}
    \end{antwoord}
\end{examenvraag}


\begin{examenvraag}
    \begin{vraag}
        Volgens Dirk De Wachter is zelfdestructie een eigenschap van onze samenleving. Hoe hebben we dit aannemelijke proberen te maken?
    \end{vraag}

    \begin{antwoord}
    \end{antwoord}
\end{examenvraag}


\begin{examenvraag}
    \begin{vraag}
        Volgens Dirk De Wachter zijn we allemaal affectlabiel. Hoe hebben we dit aannemelijk proberen te maken?
    \end{vraag}

    \begin{antwoord}
    \end{antwoord}
\end{examenvraag}


\begin{examenvraag}
    \begin{vraag}
        Volgens Dirk De Wachter hebben we allemaal last van zinloosheid en leegte. Hoe hebben we dit aannemelijk proberen te maken?
    \end{vraag}

    \begin{antwoord}
    \end{antwoord}
\end{examenvraag}


\begin{examenvraag}
    \begin{vraag}
        Volgens Dirk De Wachter leven we in een tijd die doordrenkt is van onaangepaste gressieregulatie. Hoe hebben we dit aannemelijk proberen te maken?
    \end{vraag}

    \begin{antwoord}
    \end{antwoord}
\end{examenvraag}


\begin{examenvraag}
    \begin{vraag}
        Volgens Dirk De Wachter vertonen we allemaal voorbijgaande, stressgebonden paranoïde/dissociatiesymptomen. Hoe hebben we dit aannemelijk proberen te maken?
    \end{vraag}

    \begin{antwoord}
    \end{antwoord}
\end{examenvraag}


\begin{examenvraag}
    \begin{vraag}
        Wat kenmerkt volgens Paul Verhaeghe het neoliberalisme? Wat doet het neoliberalisme met het egoïsme? Welk moreel onderscheid maakt het neoliberalisme?
    \end{vraag}

    \begin{antwoord}
    \end{antwoord}
\end{examenvraag}


\begin{examenvraag}
    \begin{vraag}
        Wat wil Paul Verhaeghe duidelijk maken met de metafoor van het “call center”?
    \end{vraag}

    \begin{antwoord}
    \end{antwoord}
\end{examenvraag}


\begin{examenvraag}
    \begin{vraag}
        Welke drie argumenten biedt Paul Verhaeghe om te onderbouwen dat het neoliberalisme negatieve gevolgen heeft?
    \end{vraag}

    \begin{antwoord}
    \end{antwoord}
\end{examenvraag}


\begin{examenvraag}
    \begin{vraag}
        Waarom is het volgens Paul Verhaeghe een misvatting te geloven dat meer markt zorgt voor meer democratie?
    \end{vraag}

    \begin{antwoord}
    \end{antwoord}
\end{examenvraag}


\begin{examenvraag}
    \begin{vraag}
        Waarom zorgt meer markt volgens Paul Verhaeghe voor meer regelgeving en procedures?
    \end{vraag}

    \begin{antwoord}
    \end{antwoord}
\end{examenvraag}


\begin{examenvraag}
    \begin{vraag}
        Hoe sluit onze bespreking van Frans De Waal (begin van thema 3) aan op de bespreking van het neoliberalisme (einde van thema 2)? Hoe hebben we met andere woorden de overgang tussen beide thema’s gemaakt?
    \end{vraag}

    \begin{antwoord}
    \end{antwoord}
\end{examenvraag}


\begin{examenvraag}
    \begin{vraag}
        Waarom kreeg het derde thema als titel “Kwetsbaar geluk” mee?
    \end{vraag}

    \begin{antwoord}
    \end{antwoord}
\end{examenvraag}


\begin{examenvraag}
    \begin{vraag}
        Wat zijn volgens Frans De Waal de drie lagen van empathie?
    \end{vraag}

    \begin{antwoord}
    \end{antwoord}
\end{examenvraag}


\begin{examenvraag}
    \begin{vraag}
        Wat is altruïsme en op welke manier is empathie hiervan het evolutionair fundament?
    \end{vraag}

    \begin{antwoord}
    \end{antwoord}
\end{examenvraag}


\begin{examenvraag}
    \begin{vraag}
        Leg uit: “Het is niet omdat biologen het voortdurend over concurrentie hebben dat ze concurrentie aanbevelen” (Frans De Waal).
    \end{vraag}

    \begin{antwoord}
    \end{antwoord}
\end{examenvraag}


\begin{examenvraag}
    \begin{vraag}
        De aanleiding voor het vertellen van de parabel van de Barmhartige Samaritaan is de vraag van een wetgeleerde. Om welke vraag gaat het hier? Waarom is dat een strikvraag? Hoe ontsnapt Jezus aan de valstrik?
    \end{vraag}

    \begin{antwoord}
    \end{antwoord}
\end{examenvraag}


\begin{examenvraag}
    \begin{vraag}
        Wat wordt bedoeld met “het lichamelijk fundament van de intermenselijke ethiek”?
    \end{vraag}

    \begin{antwoord}
    \end{antwoord}
\end{examenvraag}


\begin{examenvraag}
    \begin{vraag}
        Wat is de ethische grondervaring?
    \end{vraag}

    \begin{antwoord}
    \end{antwoord}
\end{examenvraag}


\begin{examenvraag}
    \begin{vraag}
        Waarom is de naastenliefde een gebod?
    \end{vraag}

    \begin{antwoord}
    \end{antwoord}
\end{examenvraag}


\begin{examenvraag}
    \begin{vraag}
        Op welke manier maakt het verschijnen van de noodlijdende ander mij uniek?
    \end{vraag}

    \begin{antwoord}
    \end{antwoord}
\end{examenvraag}


\begin{examenvraag}
    \begin{vraag}
        Wat zijn de kenmerken van de barmhartigheid?
    \end{vraag}

    \begin{antwoord}
    \end{antwoord}
\end{examenvraag}


\begin{examenvraag}
    \begin{vraag}
        In de hedendaagse bewerking van de parabel van de Barmhartige Samaritaan die tijdens de les getoond werd, wordt ons aangeraden iedereen lief te hebben alsof ze dood zullen zijn tegen middernacht. Hoe kunnen we dit begrijpen?
    \end{vraag}

    \begin{antwoord}
    \end{antwoord}
\end{examenvraag}


\begin{examenvraag}
    \begin{vraag}
        Bespreek de grenzen van de barmhartigheid in verwijzing naar Genesis, hoofdstuk 12.
    \end{vraag}

    \begin{antwoord}
    \end{antwoord}
\end{examenvraag}


\begin{examenvraag}
    \begin{vraag}
        Wat is de band tussen de parabel van de Barmhartige Samaritaan en Schindler’s List?
    \end{vraag}

    \begin{antwoord}
    \end{antwoord}
\end{examenvraag}


\begin{examenvraag}
    \begin{vraag}
        Waarom werden tijdens de les fragmenten uit Schindler’s List getoond? Wat hebben we geleerd uit deze fragmenten?
    \end{vraag}

    \begin{antwoord}
    \end{antwoord}
\end{examenvraag}


\begin{examenvraag}
    \begin{vraag}
        Wat is de visie van Francis Collins op de verhouding tussen geloof en wetenschap?
    \end{vraag}

    \begin{antwoord}
    \end{antwoord}
\end{examenvraag}


\begin{examenvraag}
    \begin{vraag}
        Hoe probeert Stephen Jay Gould het conflict tussen geloof en wetenschap op te lossen? Slaagt hij in zijn opzet?
    \end{vraag}

    \begin{antwoord}
    \end{antwoord}
\end{examenvraag}


\begin{examenvraag}
    \begin{vraag}
        Wat is de visie van Herman De Dijn op de verhouding tussen geloof en wetenschap? Formuleer een mogelijke kritiek op zijn visie.
    \end{vraag}

    \begin{antwoord}
    \end{antwoord}
\end{examenvraag}


\begin{examenvraag}
    \begin{vraag}
        Hoe probeert John F. Haught het conflict tussen evolutietheorie en christendom op te lossen? Formuleer een mogelijke kritiek op zijn poging.
    \end{vraag}

    \begin{antwoord}
    \end{antwoord}
\end{examenvraag}


\begin{examenvraag}
    \begin{vraag}
        Hoe denkt het differentiemodel over de verhouding tussen geloof en wetenschap?
    \end{vraag}

    \begin{antwoord}
    \end{antwoord}
\end{examenvraag}


\end{document}
