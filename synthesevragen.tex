\documentclass[main.tex]{subfiles}

\begin{document}

\section{Synthesevragen}


\begin{examenvraag}
    \begin{vraag}
        Schets de rode draad van het argument dat we doorheen de ganse cursus ontwikkeld hebben.
    \end{vraag}

    \begin{antwoord}
    \todo[inline]{TODO}
    \end{antwoord}
\end{examenvraag}


\begin{examenvraag}
    \begin{vraag}
        Tijdens de lessen hebben we het proces van individualisering gebruikt als leessleutel om de hedendaagse samenleving te begrijpen. Bespreek dit proces. Wat houdt het in? Welke oorzaken heeft het en welke gevolgen voor onszelf als individu en voor onze relaties?
    \end{vraag}

    \begin{antwoord}
    \todo[inline]{Uitleggen dat we niet perse minder met anderen omgaan}
    \end{antwoord}
\end{examenvraag}


\begin{examenvraag}
    \begin{vraag}
        Bespreek het debat tussen Paul Verhaeghe en Paul De Grauwe. Leg hierbij onder meer uit waarom het nuttig kan zijn een onderscheid te maken tussen twee betekenissen van “neoliberalisme”.
    \end{vraag}

    \begin{antwoord}
    \todo[inline]{TODO}
    \end{antwoord}
\end{examenvraag}


\begin{examenvraag}
    \begin{vraag}
        Hoe kunnen we de inzichten van Frans De Waal en de parabel van de Barmhartige Samaritaan met elkaar in verband brengen?
    \end{vraag}

    \begin{antwoord}
    \todo[inline]{TODO}
    \end{antwoord}
\end{examenvraag}


\begin{examenvraag}
    \begin{vraag}
        Eén van de kernbegrippen van de cursus was het begrip “verantwoordelijkheid”. Leg uit hoe we dat begrip in de loop van de volledige cursus ontwikkeld hebben.
    \end{vraag}

    \begin{antwoord}
    In het cursusdeel over 'identiteit' hebben we dit behandelt tijdens de bespreking van separatie.
    De relatieve keuzevrijheid van separatie impliceert een bepaalde mate van verantwoordelijkheid voor de gemaakte keuzes.
    \todo[inline]{Verder aanvullen}
    \end{antwoord}
    \begin{citaat}{Paul Verhaeghe over identiteit}
    Onze
identiteit is zonder twijfel sterk gedetermineerd, maar er is geen totale determinatie. Wij zijn
vermoedelijk de enige diersoort die op grond van reflectie over zichzelf bewuste keuzes kan maken.
Dat betekent meteen dat elk van ons ook een bepaalde mate van \textbf{verantwoordelijkheid} draagt voor die
keuzes. 
    \end{citaat}
\end{examenvraag}


\begin{examenvraag}
    \begin{vraag}
        In de cursus hebben we een onderscheid gemaakt tussen drie vormen van verantwoordelijkheid. Elke vorm van verantwoordelijkheid hebben we verbonden met een bepaalde invulling van vrijheid en een bepaalde invulling van schuld. Leg dit uit. Maak ook duidelijk hoe dit ons geholpen heeft het neoliberale begrip van vrijheid en schuld zowel ernstig te nemen alsook te nuanceren.
    \end{vraag}

    \begin{antwoord}
    \todo[inline]{TODO}
    \end{antwoord}
\end{examenvraag}


\begin{examenvraag}
    \begin{vraag}
        Wat zijn alteriteitservaringen? Waarom zijn ze belangrijk? Wat kenmerkt ze? Van welke andere, typisch hedendaagse ervaringen moeten we alteriteitservaringen onderscheiden?
    \end{vraag}

    \begin{antwoord}
    \todo[inline]{TODO}
    \end{antwoord}
\end{examenvraag}


\begin{examenvraag}
    \begin{vraag}
        Maak in je antwoord gebruik van de fragmenten uit Schindler’s List die we tijdens de les bekeken hebben.
    \end{vraag}

    \begin{antwoord}
    \todo[inline]{TODO}
    \end{antwoord}
\end{examenvraag}


\begin{examenvraag}
    \begin{vraag}
        Bespreek de verschillende modellen om de verhouding tussen geloof en wetenschap te denken die in de cursus besproken werden.
    \end{vraag}

    \begin{antwoord}
    \todo[inline]{TODO}
    \end{antwoord}
\end{examenvraag}


\begin{examenvraag}
    \begin{vraag}
        In de cursus hebben we het gehad over “resonanties” tussen geloof en wetenschap. Beantwoord hierover volgende vragen:
        \begin{itemize}
            \item Wat wordt bedoeld met “resonanties” tussen geloof en wetenschap?
            \item Op welk model om de verhouding tussen geloof en wetenschap te denken vormt het zoeken naar resonanties tussen geloof en wetenschap een aanvulling? Waarom?
            \item Op welke resonantie tussen geloof en wetenschap zijn we tijdens de collegesuitgebreid blijven stilstaan?
        \end{itemize}
    \end{vraag}

    \begin{antwoord}
    	\begin{itemize}
    		\item Wat wordt bedoeld met “resonanties” tussen geloof en wetenschap?
    			Resonantie is een muziekale term.
    			Het betekent meetrillen.
    			We leggen dit uit adh van een voorbeeld.
    			Veel gelovigen (niet de creatonisten, die zijn te hardcore) vinden dat het idee van de oerknal (afkomstig van de wetenschappers) resoneert met hun "eerst was er niets,  toen was er iets"-theorie. 
    			Ze geloven beiden andere dingen, maar ergens is er toch "common ground" waar de twee groepen wat verwantschap hebben. 
		
				Informatie uit fragment 6: Over de mogelijkheid van resonanties tussen geloof en wetenschap. (Vooral laatste halve pagina)    	

		 	\item Op welk model om de verhouding tussen geloof en wetenschap te denken vormt het zoeken naar resonanties tussen geloof en wetenschap een aanvulling? Waarom?
		 		\TODO{Verbeter?}
		 		Harmoniemodel uitgesloten. Kloofmodel is te strikt denk ik. Differentiemodel zorgt voor een dialoog op basis van constitutieve verschillen. 
		 		Het zoeken naar resonanties houdt beide kampen in ere. 
		 		Er wordt niet getracht over te gaan naar een overkoepelende eenheidstheorie, wat belangrijk is voor het differentiemodel.
		 		
		 	\item Op welke resonantie tussen geloof en wetenschap zijn we tijdens de collegesuitgebreid blijven stilstaan?
		 		\TODO{iemand naar de les geweest?}
		 		Ik vermoed dat ze the big bang en creatio ex nihilo hebben vergeleken? 	
		 	\end{itemize}		
		
    \end{antwoord}
\end{examenvraag}


\end{document}
