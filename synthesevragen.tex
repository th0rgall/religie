\documentclass[main.tex]{subfiles}

\begin{document}

\section{Synthesevragen}


\begin{examenvraag}
    \begin{vraag}
        Schets de rode draad van het argument dat we doorheen de ganse cursus ontwikkeld hebben.
    \end{vraag}

    \begin{antwoord}
    \todo[inline]{eerste vraag is enorm vaag, kweenie als hij dit bedoelde, ma lijkt mij wel het belangrijkste, een deel is gekopieerd van een andere student.}
	    Religie voor wetenschappers houdt natuurlijk in dat de verhouding tussen religie en
	    wetenschap aan bod moet komen. We hebben dan ook met dit onderwerp concreet de cursus
	    afgesloten. Toch was het onderwerp impliciet doorheen de cursus aanwezig.
	    \begin{itemize}
	    	\item We konden besluiten dat egoïsme  verankerd zit in de natuur van de mens. Doorheen de evolutie hebben de mensapen een hoger bewustzijn ontwikkeld waardoor empathie mogelijk werd.
	    	\item De voor -en nadelen van de neoliberale maatschappij werden bestudeerd.
	    		    Zo weten we bijvoorbeeld dat de verhoogde loonspanning leidt tot een negatieve evolutie van de psychosociale gezondheidsindicatoren. Zo kunnen we objectief besluiten dat een hogere loonspanning
	    		    een slechte zaak is.
	   		\item Nog een punt waarbij wetenschap impliciet aan bod is gekomen is bij identiteit.
	   			    Aan de hand van tweelingen zien we dat mensen met identiek genetisch materiaal
	   			    toch verschillende persoonlijkheden (identiteiten) ontwikkelen als ze in verschillende omgevingen
	   			    opgroeien. Zo kan men objectief besluiten dat het erfelijk materiaal niet uitsluitend
	   			    verantwoordelijk
	   			    is voor de vorming van onze identiteit.
	   		\item ...
	    \end{itemize}
	    Conclusie: men gebruikt wetenschap om de religie te helpen sturen. Religie mag
	    wetenschappelijke
	    ontwikkelingen niet negeren, maar moet ze gebruiken. $\Rightarrow$ differentiemodel
    \end{antwoord}
\end{examenvraag}


\begin{examenvraag}
    \begin{vraag}
        Tijdens de lessen hebben we het proces van individualisering gebruikt als leessleutel om de hedendaagse samenleving te begrijpen. Bespreek dit proces. Wat houdt het in? Welke oorzaken heeft het en welke gevolgen voor onszelf als individu en voor onze relaties?
    \end{vraag}

    \begin{antwoord}
    Individualisering is een twee-staps proces:‭ ‬het bestaat uit detraditionalisering en vrijmaking.‭ 
    Detraditionalisering wijst erop dat we de vooropgelegde normen en waarden reflecteren en 
    relativeren.‭ ‬Dit betekent niet dat tradities verdwijnen maar wanneer we ons eraan willen houden 
    is dit onze eigen,‭ ‬vrije keuze.‭ ‬Bovendien verbreedt het spectrum aan keuzes enorm door 
    vrijmaking.‭ ‬Wat je later als mens wordt,‭ ‬hangt steeds minder af van je omgeving en wordt je 
    eigen keuze.‭ ‬Dit brengt wel met zich mee dat je verplicht bent om te kiezen,‭ 
    ‬identiteitsconstructie wordt een plicht.‭ ‬Dit leidt ons tot een meritocratie waarbij je krijgt 
    wat je verdient.‭ ‬Dit zorgt er echter voor dat de lat steeds hoger wordt gelegd wat ons leidt in 
    een prestatiemaatschappij.‭ ‬Door individualisering is ook groepsvorming een‭ ‬keuze.‭ ‬Je kiest je 
    omgeving bewust op basis van subjectieve factoren en waar je je niet thuis voelt,‭ ‬blijf je 
    niet.‭ ‬Je socialiteit is geen vast begrip maar een vlottend iets.‭ ‬Bovendien beantwoorden 
    relaties niet langer aan vooropgestelde rolverwachtingen.‭ ‬Onderlinge relaties worden steeds 
    meer een onderling afgesproken begrip.‭ ‬Een geïndividualiseerde leefwereld hoeft dus niet 
    egoïstisch te zijn.‭ ‬Er bestaan nog steeds overeenkomstige waarden en normen maar je leefwereld 
    is een plaats van overleg.
    \end{antwoord}
\end{examenvraag}


\begin{examenvraag}
    \begin{vraag}
        Bespreek het debat tussen Paul Verhaeghe en Paul De Grauwe. Leg hierbij onder meer uit waarom het nuttig kan zijn een onderscheid te maken tussen twee betekenissen van “neoliberalisme”.
    \end{vraag}

    \begin{antwoord}
	    \textbf{Paul Verhaege} omschrijft neoliberalisme als een nieuw en dwingend narratief dat onze
	    moderne maatschapij typeert. Het kernidee is dat efficiëntie het hoogste goed is. Efficiëntie
	    verkrijgt men door concurrentie op te zetten tussen mensen (werknemers). Typisch gevolg hiervan
	    is dat egoïsme beschreven word als de natuur van de mens en zelfs als een goede zaak. Ander gevolg
	    is dat mensen verplicht worden om 'het te maken', je bent zelf verantwoordelijk voor je succes.
	    Paul Verhaege is ook sterk tegen dit neoliberalisme en omschrijft hoe het (deels) de oorzaak is van
	    stijgende loonspanning, een gevoel van onbehagen en rampzalige ecologische gevolgen. Bovendien
	    zorgt
	    het neoliberalisme voor meer regelgeving.\\\\
	    \textbf{Paul de Grauwe} is zelf economist en probeert het standpunt van Verhaege te relativeren:
	    Neoliberalisme verschilt niet zo fel van wat we al lang kennen. Hij stelt dat de vrije markt
	    nog steeds het beste systeem is om de economische verhouding tussen mensen te regelen.
	    (maak hierbij wel de opmerking dat hij niet pleit voor volledig vrije mark: laissez faire, maar voor
	    vrije markt met regelgeving).\\\\
	    Dus ter samenvatting, neoliberalisme is volgens
		\begin{itemize}
			\item Verhaege: Liberalisme waarbij de politiek ondergeschikt is aan de markt.
			\item De Grauwe: De samenleving moet zo georganiseerd worden dat iedereen zoveel mogelijk vrijheid
				    heeft.
		\end{itemize}
	    
	    
	    $\rightarrow$ 2 definities
    \end{antwoord}
\end{examenvraag}


\begin{examenvraag}
    \begin{vraag}
        Hoe kunnen we de inzichten van Frans De Waal en de parabel van de Barmhartige Samaritaan met elkaar in verband brengen?
    \end{vraag}

    \begin{antwoord}
    	De parabel van de \textbf{barmhartige Samaritaan} probeert uit te leggen dat mensen barmhartigheid
    	moeten tonen voor elkaar, Hun verantwoordelijkheid in de 2e persoon moeten opnemen. Maar kan men dit wel verwachten van de mens? Want volgens (Paul Verhaege's
    	versie van) het neoliberalisme is de mens van nature egoïstisch en is dit ook een goede zaak.
    	Mensen zullen dus van nature geen barmhartigheid tonen en hoeven dit ook niet te doen...\\\\
    	\textbf{Frans de Waal} probeert te duiden dat dit helemaal niet waar is. De mens heeft een 
    	natuurlijk vermogen tot empathie. Dit komt ook voor bij chimpansees, wat aantoont dat
    	het een valabele overlevingsstrategie is. Empathie is dus wél een goede zaak.
    \end{antwoord}
\end{examenvraag}


\begin{examenvraag}
    \begin{vraag}
        Eén van de kernbegrippen van de cursus was het begrip “verantwoordelijkheid”. Leg uit hoe we dat begrip in de loop van de volledige cursus ontwikkeld hebben.
    \end{vraag}

    \begin{antwoord}
	Bij de vorming van onze identiteit speelt de omgeving een doorslaggevende factor.‭ ‬Zijn we 
	daarom niet verantwoordelijk voor onze identiteit‭? ‬Neen,‭ ‬als mens zijn we in staat onze eigen 
	omgeving te wijzigen.‭ ‬Dit komt door één van de‭ ‬twee processen waarmee onze identiteit wordt 
	opgebouwd:‭ ‬separatie.‭ ‬Hierdoor heeft onze individuele identiteit een uniek karakter.‭ ‬Wij zijn 
	immers in staat o.b.v.‭ ‬zelfreflectie over onze eigen keuzes te bewust oordelen.‭ ‬Hierdoor dragen 
	we erover een verantwoordelijkheid.‭
	‬Het neoliberalisme stelt dat we zelf verantwoordelijk zijn om‭ ‬‘het‭’‬ te maken.‭ ‬Want vandaag is 
	alles een kwestie van keuze en eigen inspanning.‭ ‬Wie faalt,‭ ‬heeft dit louter aan zichzelf te 
	danken en is een profiteur.‭ ‬Zo werkt het neoliberalisme in op het schuldgevoel van de mens die‭ 
	‬‘het‭’‬ helaas niet heeft kunnen maken.\\
	We hebben ook geleerd om dat we onze verantwoordelijkheid moeten opnemen op 3 verschillende manieren:
	\begin{itemize}
		\item Verantwoordelijkheid in de \textbf{1e} persoon: Verantwoordelijkheidsgevoel voor je eigen ontwikkeling. Zorgen dat je je handhaaft in het bestaan.
		Een beetje egocentrisme, om je eigen doelen(verhaal) na te streven.
		\item Verantwoordelijkheid in de \textbf{2e} persoon: Verantwoordelijkheidsgevoel voor een ander (in nood). Begaan zijn met de emotionele/fysieke welgesteldheid
		van een medemens. Letterlijk: mijn raakbaarheid voor de lijdende ander. Als je deze verantwoordelijkheid
		opneemt spreekt men van barmhartigheid.
		\item Verantwoordelijkheid in de \textbf{3e} persoon: Verantwoordelijkheidsgevoel voor de derde, die niet aanwezig is in tijd en ruimte. Voor de volgende
		generaties.
	\end{itemize}
    \end{antwoord}
    \begin{citaat}{Paul Verhaeghe over identiteit}
    Onze
identiteit is zonder twijfel sterk gedetermineerd, maar er is geen totale determinatie. Wij zijn
vermoedelijk de enige diersoort die op grond van reflectie over zichzelf bewuste keuzes kan maken.
Dat betekent meteen dat elk van ons ook een bepaalde mate van \textbf{verantwoordelijkheid} draagt voor die
keuzes. 
    \end{citaat}
\end{examenvraag}


\begin{examenvraag}
    \begin{vraag}
        In de cursus hebben we een onderscheid gemaakt tussen drie vormen van verantwoordelijkheid. Elke vorm van verantwoordelijkheid hebben we verbonden met een bepaalde invulling van vrijheid en een bepaalde invulling van schuld. Leg dit uit. Maak ook duidelijk hoe dit ons geholpen heeft het neoliberale begrip van vrijheid en schuld zowel ernstig te nemen alsook te nuanceren.
    \end{vraag}

    \begin{antwoord}
    \begin{itemize}
    		\item Verantwoordelijkheid in de \textbf{1e} persoon: Verantwoordelijkheidsgevoel voor je eigen ontwikkeling. Zorgen dat je je handhaaft in het bestaan. Een beetje egocentrisme, om je eigen doelen(verhaal) na te streven. Het tegendeel van deze vrijheid = verslaving en obsessionele afhankelijkheid.\\
    		Ernstige opvatting: Je bent verantwoordelijk voor je leven, je moet het in handen nemen.\\
    		Nuancering: 
    		\begin{itemize}
    			\item Mensen zijn niet steeds vrij, maar moeten vaak nog vrij gemaakt worden.
    			\item Mislukken is niet uitsluitend persoonlijke schuld, maar inherent verbonden met onze eindigheid
    			
    		\end{itemize}
    		 
    		
    		\item Verantwoordelijkheid in de \textbf{2e} persoon: Verantwoordelijkheidsgevoel voor een ander (in nood). Begaan zijn met de emotionele/fysieke welgesteldheid
    		van een medemens. Letterlijk: mijn raakbaarheid voor de lijdende ander. Als je deze verantwoordelijkheid
    		opneemt spreekt men van barmhartigheid.\\
    		Ernstige opvatting: We moeten streven naar oneindige barmhartigheid.\\
    		Nuancering: Er moet een zekere grens worden getrokken:
    		\begin{itemize}
    			\item De eigen persoonswaarde en integriteit moet blijven
    			\item De verantwoordelijkheid in de 3de persoon mag niet ten koste gaan van een noodlijdende ander.
    		\end{itemize}
    		
    		
    		\item Verantwoordelijkheid in de \textbf{3e} persoon: Verantwoordelijkheidsgevoel voor de derde, die niet aanwezig is in tijd en ruimte. Voor de volgende generaties.
    		Deze verantwoordelijkheid wordt echter begrensd door onze vrees. Wat ons beangstigd wijst ons wat echt belangrijk is, wat er op het spel staat en waarop we moeten letten.\\
    		Niet: "Wat zouden we graag hebben?"
    	\end{itemize}
    \end{antwoord}
\end{examenvraag}


\begin{examenvraag}
    \begin{vraag}
        Wat zijn alteriteitservaringen? Waarom zijn ze belangrijk? Wat kenmerkt ze? Van welke andere, typisch hedendaagse ervaringen moeten we alteriteitservaringen onderscheiden?
    \end{vraag}

    \begin{antwoord}
    \todo[inline]{TODO}
    \end{antwoord}
\end{examenvraag}


\begin{examenvraag}
    \begin{vraag}
        Maak in je antwoord gebruik van de fragmenten uit Schindler’s List die we tijdens de les bekeken hebben.
    \end{vraag}

    \begin{antwoord}
    \todo[inline]{TODO}
    \end{antwoord}
\end{examenvraag}


\begin{examenvraag}
    \begin{vraag}
        Bespreek de verschillende modellen om de verhouding tussen geloof en wetenschap te denken die in de cursus besproken werden.
    \end{vraag}

    \begin{antwoord}
		\begin{itemize}
			\item \textbf{Conflict model:} Hierbij botsen geloof en wetenschap tegen elkaar en moet er dus een keuze gemaakt worden. Ze kunnen gewoonweg niet samenleven.
			\item \textbf{Kloofmodel:} Hierbij verschillen geloof en wetenschap zo grondig van elkaar dat ze eenvoudigweg niet met elkaar in conflict kunnen komen. Elk van hen heeft zijn eigen taal en taalregister. Wetenschappen proberen de realiteit te manipuleren/verklaren, terwijl zingeving, etiek en religie levensvragen proberen te beantwoorden.
			\item \textbf{Harmoniemodel:} Hierbij kunnen geloof en wetenschap verzoend worden met elkaar. Ze leven samen als een geheel, een eenheid.
			\item \textbf{Differentiemodel:} Hierbij is er geen conflict, scheiding of integratie, maar een 'open' dialoog op basis van het constituieve verschil. De gelovige ziet de wetenschap als een uitdaging voor het eigen gelovige verhaal. Wetenschap is niet de vijand, niet de concurrent, ook niet de vreemde die voor het christelijke geloof volstrekt irrelevant is, maar een partner in de zoektocht naar een beter begrijpen van mens en wereld, ook spreken beide een andere taal en behoren de resultaten van de wetenschappelijke zoektocht tot een ander taalregister.
			 
		\end{itemize}
    \end{antwoord}
\end{examenvraag}


\begin{examenvraag}
    \begin{vraag}
        In de cursus hebben we het gehad over “resonanties” tussen geloof en wetenschap. Beantwoord hierover volgende vragen:
        \begin{itemize}
            \item Wat wordt bedoeld met “resonanties” tussen geloof en wetenschap?
            \item Op welk model om de verhouding tussen geloof en wetenschap te denken vormt het zoeken naar resonanties tussen geloof en wetenschap een aanvulling? Waarom?
            \item Op welke resonantie tussen geloof en wetenschap zijn we tijdens de collegesuitgebreid blijven stilstaan?
        \end{itemize}
    \end{vraag}

    \begin{antwoord}
    	\begin{itemize}
    		\item Wat wordt bedoeld met “resonanties” tussen geloof en wetenschap?
    			Resonantie is een muziekale term.
    			Het betekent meetrillen.
    			We leggen dit uit adh van een voorbeeld.
    			Veel gelovigen (niet de creationisten, die zijn te hardcore) vinden dat het idee van de oerknal (afkomstig van de wetenschappers) resoneert met hun 'eerst was er niets,  toen was er iets'-theorie. 
    			Ze geloven beiden andere dingen, maar ergens is er toch "common ground" waar de twee groepen wat verwantschap hebben. 
		
				Informatie uit fragment 6: Over de mogelijkheid van resonanties tussen geloof en wetenschap. (Vooral laatste halve pagina)    	

		 	\item Op welk model om de verhouding tussen geloof en wetenschap te denken vormt het zoeken naar resonanties tussen geloof en wetenschap een aanvulling? Waarom?
		 		\TODO{Verbeter?}
		 		Harmoniemodel uitgesloten. Kloofmodel is te strikt denk ik. Differentiemodel zorgt voor een dialoog op basis van constitutieve verschillen. 
		 		Het zoeken naar resonanties houdt beide kampen in ere. 
		 		Er wordt niet getracht over te gaan naar een overkoepelende eenheidstheorie, wat belangrijk is voor het differentiemodel.
		 		
		 	\item Op welke resonantie tussen geloof en wetenschap zijn we tijdens de colleges uitgebreid blijven stilstaan?
		 		\TODO{iemand naar de les geweest?}
		 		Ik vermoed dat ze the big bang en creatio ex nihilo hebben vergeleken? 	
		 	\end{itemize}		
		
    \end{antwoord}
\end{examenvraag}


\end{document}
