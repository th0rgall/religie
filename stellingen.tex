\documentclass[main.tex]{subfiles}

\begin{document}
\section{Stellingen}

\subsection{Volgens Paul Verhaeghe heeft onze identiteit een biologisch-evolutionair fundament.}
\subsection{Volgens Paul Verhaeghe is onze identiteit een sociale constructie en heeft die bijgevolg geen biologisch fundament.}
\subsection{Volgens Paul Verhaeghe bewijst adoptie dat onze identiteit geen biologisch-evolutionair fundament heeft.}
\subsection{Hoe we denken over onszelf (m.a.w. ons zelfbeeld) is een constructie vanuit de omgeving.}
\subsection{Detraditionalisering betekent dat tradities verdwijnen.}
\subsection{Individualisering betekent dat we met zijn allen egoïstischer worden.}
\subsection{Individualisering betekent dat sociale ongelijkheid verdwijnt.}
\subsection{Individualisering betekent in Paul Verhaeghes termen: meer mogelijkheid tot separatie.}
\subsection{Volgens Paul Verhaeghe leidt meer vrije markt tot meer vrijheid.}
\subsection{Volgens Paul Verhaeghe gaan meer vrije markt en meer democratie hand in hand.}
\subsection{Empathie vraagt om sterk ontwikkelde cognitieve vermogens en komt bijgevolg slechts bij weinig dieren voor.}
\subsection{Gerichte hulp is wijdverbreid in het dierenrijk en moet bijgevolg reeds vroeg in de evolutie van het leven ontstaan zijn.}
\subsection{Volgens de cursus is een gezonde dosis egocentrisme noodzakelijk.}
\subsection{Het feit dat de priester en de leviet in de parabel van de Barmhartige Samaritaan met een wijde boog om de gewonde man heen lopen maakt duidelijk dat ze hem niet hebben zien liggen.}
\subsection{De verschijning van de lijdende ander leidt tot het einde van mijn vrijheid.}
\subsection{Naastenliefde betekent dat je iedereen even sympathiek moet vinden.}
\subsection{Volgens de cursus is het absurd om van de naastenliefde een gebod te maken.}
\subsection{De naastenliefde is een gebod omdat ze tegennatuurlijk is.}
\subsection{De parabel van de Barmhartige Samaritaan leert ons alles over de ethische verhouding tussen mensen.}
\subsection{De parabel van de Barmhartige Samaritaan presenteert de Samaritaan waar de parabel naar genoemd is als een model van totale barmhartigheid.}
\subsection{Gelovige wetenschappers zijn volgens het koofmodel eigenlijk schizofreen want je kan niet tegelijk een goede wetenschapper en een goede gelovige zijn.}
\subsection{Wie vandaag een harmonie tussen geloof en natuurwetenschap nastreeft, bevordert volgens Taede Smedes eigenlijk het conflict tussen beide.}
\subsection{Volgens het differentiemodel moeten gelovigen rekening houden met de resultaten van wetenschappelijk onderzoek.}
\subsection{Volgens het differentiemodel zijn de resultaten van wetenschappelijk onderzoek irrelevant voor het geloof.}
\subsection{Taede Smedes is een aanhanger van het kloofmodel want hij verdedigt een complete “boedelscheiding” tussen beide.}
\end{document}
