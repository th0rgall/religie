\documentclass[main.tex]{subfiles}

\begin{document}
\section{Stellingen: ``Waar of onwaar? Geef telkens een beknopte motivatie''}

\begin{examenvraag}
    \begin{stelling}
        Volgens Paul Verhaeghe heeft onze identiteit een biologisch-evolutionair fundament.
    \end{stelling}

    \begin{stelling-antwoord}{Waar}
        Indentiteit is een sociale constructie bovenop een evolutionair-biologisch fundament. Deze bestaat volgens Verhaeghe uit drie evolutionair bepaalde kenmerken:
        \begin{itemize}
        \item De mens behoort tot de sociale diersoorten
        \item De mens is een hiërarchische soort
        \item De mens vertoont 2 verschillende gedragsclusters
        \end{itemize}
        \begin{citaat}{Paul Verhaeghe over indentiteit}
            Identiteit is grotendeels een sociale constructie, maar bouwen doe je niet in het luchtledige.
            Het is een constructie bovenop een evolutionair-biologisch fundament en door onze focus op genen en 2 chromosomen houden we daar vandaag de dag te weinig rekening mee.
        \end{citaat}
    \end{stelling-antwoord}
\end{examenvraag}


\begin{examenvraag}
    \begin{stelling}
        Volgens Paul Verhaeghe is onze identiteit een sociale constructie en heeft die bijgevolg geen biologisch fundament.
    \end{stelling}

    \begin{stelling-antwoord}{Onwaar}
        Ze heeft zeker een biologisch fundament, daarop wordt gebouwdt tijdens de sociale constructie. Deze bestaat volgens Verhaeghe uit drie evolutionair bepaalde kenmerken:
        \begin{itemize}
        \item De mens behoort tot de sociale diersoorten
        \item De mens is een hiërarchische soort
        \item De mens vertoont 2 verschillende gedragsclusters
        \end{itemize}
        \begin{citaat}{Paul Verhaeghe over indentiteit}
            Identiteit is grotendeels een sociale constructie, maar bouwen doe je niet in het luchtledige.
            Het is een constructie bovenop een evolutionair-biologisch fundament en door onze focus op genen en 2 chromosomen houden we daar vandaag de dag te weinig rekening mee.
        \end{citaat}
    \end{stelling-antwoord}
\end{examenvraag}


\begin{examenvraag}
    \begin{stelling}
        Volgens Paul Verhaeghe bewijst adoptie dat onze identiteit geen biologisch-evolutionair fundament heeft.
    \end{stelling}

    \begin{stelling-antwoord}{Onwaar}
        Paul Verhaeghe gebruikt adoptie als argument om de twee grote grote misvattingen rond identiteit aan te tonen. Het toont er mee aan dat identiteit niet biologisch gedetermineerd is en dat ze niet onveranderlijk is.
        \begin{citaat}{Paul Verhaeghe over indentiteit}
            Mijn identiteit is een constructie van dergelijke verhoudingen tegenover de ander.
            Het woord constructie impliceert dat ik iemand anders had kunnen worden, mocht het constructieproces anders verlopen zijn.
            Het meest overtuigende bewijs daarvoor is adoptie.
        \end{citaat}
    \end{stelling-antwoord}
\end{examenvraag}


\begin{examenvraag}
    \begin{stelling}
        Hoe we denken over onszelf (m.a.w. ons zelfbeeld) is een constructie vanuit de omgeving.
    \end{stelling}

    \begin{stelling-antwoord}{Waar}
        Het zelfbeeld wordt gevormd, van kinds af aan, door de omgang van anderen met ons.
        \begin{citaat}{Paul Verhaeghe over identiteit}
            Op grond van wat ik te horen gekregen heb tijdens mijn identiteitsuitbouw, ben ik zelfzeker, stap ik vol vertrouwen de ander tegemoet, ga ik er automatisch van uit dat ik beter ben dan die ander.
            Of: ben ik angstig, beschaamd over mezelf, overtuigd dat de ander boos is op mij, mij maar niks vindt en probeer ik op voorhand al te ontsnappen aan de dreiging die ik meen te voelen. In psychiatrisch vakjargon: een hoge sociale angst.
        \end{citaat}
    \end{stelling-antwoord}
\end{examenvraag}


\begin{examenvraag}
    \begin{stelling}
        Detraditionalisering betekent dat tradities verdwijnen.
    \end{stelling}

	\begin{stelling-antwoord}{Onwaar}
       De term betekent dat tradities hun dwingend karakter verliezen. Het worden mogelijkheden in plaats van verplichtingen.
       \begin{citaat}{Slides: ``Wat kenmerkt onze maatschappij? (18)''}
			Opmerking: detraditionalisering betekent niet dat tradities verdwijnen: “sociaal dwingende opties” $\rightarrow$ “mogelijke mogelijkheden”
        \end{citaat}
    \end{stelling-antwoord}
\end{examenvraag}
	

\begin{examenvraag}
    \begin{stelling}
        Individualisering betekent dat we met zijn allen egoïstischer worden.
    \end{stelling}

    \begin{antwoord}
    \end{antwoord}
\end{examenvraag}


\begin{examenvraag}
    \begin{stelling}
        Individualisering betekent dat sociale ongelijkheid verdwijnt.
    \end{stelling}

    \begin{antwoord}
    \end{antwoord}
\end{examenvraag}


\begin{examenvraag}
    \begin{stelling}
        Individualisering betekent in Paul Verhaeghes termen: meer mogelijkheid tot separatie.
    \end{stelling}

    \begin{antwoord}
    \end{antwoord}
\end{examenvraag}


\begin{examenvraag}
    \begin{stelling}
        Volgens Paul Verhaeghe leidt meer vrije markt tot meer vrijheid.
    \end{stelling}

    \begin{antwoord}
    \end{antwoord}
\end{examenvraag}


\begin{examenvraag}
    \begin{stelling}
        Volgens Paul Verhaeghe gaan meer vrije markt en meer democratie hand in hand.
    \end{stelling}

    \begin{antwoord}
    \end{antwoord}
\end{examenvraag}


\begin{examenvraag}
    \begin{stelling}
        Empathie vraagt om sterk ontwikkelde cognitieve vermogens en komt bijgevolg slechts bij weinig dieren voor.
    \end{stelling}

    \begin{antwoord}
    \end{antwoord}
\end{examenvraag}


\begin{examenvraag}
    \begin{stelling}
        Gerichte hulp is wijdverbreid in het dierenrijk en moet bijgevolg reeds vroeg in de evolutie van het leven ontstaan zijn.
    \end{stelling}

    \begin{antwoord}
    \end{antwoord}
\end{examenvraag}


\begin{examenvraag}
    \begin{stelling}
        Volgens de cursus is een gezonde dosis egocentrisme noodzakelijk.
    \end{stelling}

    \begin{antwoord}
    \end{antwoord}
\end{examenvraag}


\begin{examenvraag}
    \begin{stelling}
        Het feit dat de priester en de leviet in de parabel van de Barmhartige Samaritaan met een wijde boog om de gewonde man heen lopen maakt duidelijk dat ze hem niet hebben zien liggen.
    \end{stelling}

    \begin{antwoord}
    \end{antwoord}
\end{examenvraag}


\begin{examenvraag}
    \begin{stelling}
        De verschijning van de lijdende ander leidt tot het einde van mijn vrijheid.
    \end{stelling}

    \begin{antwoord}
    \end{antwoord}
\end{examenvraag}


\begin{examenvraag}
    \begin{stelling}
        Naastenliefde betekent dat je iedereen even sympathiek moet vinden.
    \end{stelling}

    \begin{antwoord}
    \end{antwoord}
\end{examenvraag}


\begin{examenvraag}
    \begin{stelling}
        Volgens de cursus is het absurd om van de naastenliefde een gebod te maken.
    \end{stelling}

    \begin{antwoord}
    \end{antwoord}
\end{examenvraag}


\begin{examenvraag}
    \begin{stelling}
        De naastenliefde is een gebod omdat ze tegennatuurlijk is.
    \end{stelling}

    \begin{antwoord}
    \end{antwoord}
\end{examenvraag}


\begin{examenvraag}
    \begin{stelling}
        De parabel van de Barmhartige Samaritaan leert ons alles over de ethische verhouding tussen mensen.
    \end{stelling}

    \begin{antwoord}
    \end{antwoord}
\end{examenvraag}


\begin{examenvraag}
    \begin{stelling}
        De parabel van de Barmhartige Samaritaan presenteert de Samaritaan waar de parabel naar genoemd is als een model van totale barmhartigheid.
    \end{stelling}

    \begin{antwoord}
    \end{antwoord}
\end{examenvraag}


\begin{examenvraag}
    \begin{stelling}
        Gelovige wetenschappers zijn volgens het koofmodel eigenlijk schizofreen want je kan niet tegelijk een goede wetenschapper en een goede gelovige zijn.
    \end{stelling}

    \begin{antwoord}
    \end{antwoord}
\end{examenvraag}


\begin{examenvraag}
    \begin{stelling}
        Wie vandaag een harmonie tussen geloof en natuurwetenschap nastreeft, bevordert volgens Taede Smedes eigenlijk het conflict tussen beide.
    \end{stelling}

    \begin{antwoord}
    \end{antwoord}
\end{examenvraag}


\begin{examenvraag}
    \begin{stelling}
        Volgens het differentiemodel moeten gelovigen rekening houden met de resultaten van wetenschappelijk onderzoek.
    \end{stelling}

    \begin{antwoord}
    \end{antwoord}
\end{examenvraag}


\begin{examenvraag}
    \begin{stelling}
        Volgens het differentiemodel zijn de resultaten van wetenschappelijk onderzoek irrelevant voor het geloof.
    \end{stelling}

    \begin{antwoord}
    \end{antwoord}
\end{examenvraag}


\begin{examenvraag}
    \begin{stelling}
        Taede Smedes is een aanhanger van het kloofmodel want hij verdedigt een complete “boedelscheiding” tussen beide.
    \end{stelling}

    \begin{antwoord}
    \end{antwoord}
\end{examenvraag}


\end{document}
