\documentclass[main.tex]{subfiles}

\begin{document}
\section{Stellingen: ``Waar of onwaar? Geef telkens een beknopte motivatie''}

\begin{examenvraag}
    \begin{stelling}
        Volgens Paul Verhaeghe heeft onze identiteit een biologisch-evolutionair fundament.
    \end{stelling}

    \begin{stelling-antwoord}{Waar}
        Indentiteit is een sociale constructie bovenop een evolutionair-biologisch fundament. Deze bestaat volgens Verhaeghe uit drie evolutionair bepaalde kenmerken:
        \begin{itemize}
        \item De mens behoort tot de sociale diersoorten
        \item De mens is een hiërarchische soort
        \item De mens vertoont 2 verschillende gedragsclusters
        \end{itemize}
        \begin{citaat}{Paul Verhaeghe over indentiteit}
            Identiteit is grotendeels een sociale constructie, maar bouwen doe je niet in het luchtledige.
            Het is een constructie bovenop een evolutionair-biologisch fundament en door onze focus op genen en 2 chromosomen houden we daar vandaag de dag te weinig rekening mee.
        \end{citaat}
    \end{stelling-antwoord}
\end{examenvraag}


\begin{examenvraag}
    \begin{stelling}
        Volgens Paul Verhaeghe is onze identiteit een sociale constructie en heeft die bijgevolg geen biologisch fundament.
    \end{stelling}

    \begin{stelling-antwoord}{Onwaar}
        Ze heeft zeker een biologisch fundament, daarop wordt gebouwdt tijdens de sociale constructie. Deze bestaat volgens Verhaeghe uit drie evolutionair bepaalde kenmerken:
        \begin{itemize}
        \item De mens behoort tot de sociale diersoorten
        \item De mens is een hiërarchische soort
        \item De mens vertoont 2 verschillende gedragsclusters
        \end{itemize}
        \begin{citaat}{Paul Verhaeghe over indentiteit}
            Identiteit is grotendeels een sociale constructie, maar bouwen doe je niet in het luchtledige.
            Het is een constructie bovenop een evolutionair-biologisch fundament en door onze focus op genen en 2 chromosomen houden we daar vandaag de dag te weinig rekening mee.
        \end{citaat}
    \end{stelling-antwoord}
\end{examenvraag}


\begin{examenvraag}
    \begin{stelling}
        Volgens Paul Verhaeghe bewijst adoptie dat onze identiteit geen biologisch-evolutionair fundament heeft.
    \end{stelling}

    \begin{stelling-antwoord}{Onwaar}
        Paul Verhaeghe gebruikt adoptie als argument om de twee grote grote misvattingen rond identiteit aan te tonen. Het toont er mee aan dat identiteit niet biologisch gedetermineerd is en dat ze niet onveranderlijk is.
        \begin{citaat}{Paul Verhaeghe over indentiteit}
            Mijn identiteit is een constructie van dergelijke verhoudingen tegenover de ander.
            Het woord constructie impliceert dat ik iemand anders had kunnen worden, mocht het constructieproces anders verlopen zijn.
            Het meest overtuigende bewijs daarvoor is adoptie.
        \end{citaat}
    \end{stelling-antwoord}
\end{examenvraag}


\begin{examenvraag}
    \begin{stelling}
        Hoe we denken over onszelf (m.a.w. ons zelfbeeld) is een constructie vanuit de omgeving.
    \end{stelling}

    \begin{stelling-antwoord}{Waar}
        Het zelfbeeld wordt gevormd, van kinds af aan, door de omgang van anderen met ons.
        \begin{citaat}{Paul Verhaeghe over identiteit}
            Op grond van wat ik te horen gekregen heb tijdens mijn identiteitsuitbouw, ben ik zelfzeker, stap ik vol vertrouwen de ander tegemoet, ga ik er automatisch van uit dat ik beter ben dan die ander.
            Of: ben ik angstig, beschaamd over mezelf, overtuigd dat de ander boos is op mij, mij maar niks vindt en probeer ik op voorhand al te ontsnappen aan de dreiging die ik meen te voelen. In psychiatrisch vakjargon: een hoge sociale angst.
        \end{citaat}
    \end{stelling-antwoord}
\end{examenvraag}


\begin{examenvraag}
    \begin{stelling}
        Detraditionalisering betekent dat tradities verdwijnen.
    \end{stelling}

    \begin{stelling-antwoord}{Onwaar}
        De term betekent dat tradities hun dwingend karakter verliezen. Het worden mogelijkheden in plaats van verplichtingen.
        \begin{citaat}{Slides: ``Wat kenmerkt onze maatschappij? (18)''}
            Opmerking: detraditionalisering betekent niet dat tradities verdwijnen: “sociaal dwingende opties” $\rightarrow$ “mogelijke mogelijkheden”
        \end{citaat}
        \begin{citaat}{Slides: ``Wat kenmerkt onze maatschappij (15)''}
            Detraditionalisering = “de relativering van alle traditionele normen en waarden”; “collectief gedragen opvattingen”, alledaagse handelingen en omgangsvormen verliezen hun dwingend karakter.
        \end{citaat}
        \begin{citaat}{In de greep van ``de moderne tijd'' (p. 65)}
            Het naoorlogse individualiseringsproces bezit kortom een dubbel gezicht.
            Voreerst behelst het een ingrijpende `de-traditionalisering' (Beck) van de leefwereld.
            Aan deze `vrijmakingsdimensie' beantwoordt een relatief sterk toegenomen individuele beslissingsvrijheid inzake waarden, normen en alledaagse handelingen (kleding, vrije tijd, voeding,...).
        \end{citaat}
    \end{stelling-antwoord}
\end{examenvraag}
	

\begin{examenvraag}
    \begin{stelling}
        Individualisering betekent dat we met zijn allen egoïstischer worden.
    \end{stelling}

    \begin{stelling-antwoord}{Onwaar}
        Individualisering betekent:
        \begin{enumerate}
            \item Relativering van traditionele normen en waarden.
            \item Keuzevrijheid in overtuigingen.
        \end{enumerate}
        \begin{citaat}{Slides: ``Wat kenmerkt onze maatschappij? (28)''}
            Individualisering $\neq$ toenemend ego\"isme, individualisme of een atomisering van de leefwereld.
        \end{citaat}
        \begin{citaat}{In de greep van ``de moderne tijd'' (p. 65)}
            Een ge\"individualiseerde maatschappij (leefwereld - RL) is niet een maatschappij van alleenstaanden, van individualisten of van hebberige ego\"isten.
            Individualisering...(betekent) in de eerste plaats de relativering van alle traditionele normen en waarden; in de tweede plaats staat zij voor de overtuiging dat er in de wereld vele mogelijkheden openliggen en dat men in principe zelf moet kunnen kiezen welke men wil gebruiken.
        \end{citaat}
         \begin{citaat}{In de greep van ``de moderne tijd'' (p. 67)}
            Individualisering is noch synoniem met `ego\"isering', noch met een 
            groeiende atomisering van de leefwereld.
        \end{citaat}
    \end{stelling-antwoord}
\end{examenvraag}


\begin{examenvraag}
    \begin{stelling}
        Individualisering betekent dat sociale ongelijkheid verdwijnt.
    \end{stelling}

    \begin{stelling-antwoord}{Onwaar}
        Sociale ongelijkheid verdwijnt niet, maar uit zich niet meer in de vorm van klassenculturen.
        In plaats daarvan uit sociale ongelijkheid zich nu meer in consumptiepatronen.
        \begin{citaat}{In de greep van ``de moderne tijd'' (p. 67)}
            Over de ondertussen `ge\"individualiseerde keuzen' vallen nog wel degelijk steeds zekere algemene uitspraken te doen.
            In de eerste plaats blijven individuele voor- en afkeuren aan materi\"ele en scolaire (on)mogelijkheden gebonden.
            De sociale ongelijkheid veruitwendigt zich daarom nog altijd in uiteenlopende consumptie) en participatie-patronen.
            [...]
            De `objectieve klassen' blijven verder bestaan maar de klassenculturen 
            verdwijnen, zodat aan de sociologische klassenconstructies geen duurzame sociale collectiviteiten of `wij-gevoelens' meer beantwoorden.
        \end{citaat}
        \begin{citaat}{Slides: ``Wat kenmerkt onze maatschappij? (31)''}
            ``Identiteit is er voor wie het zich kan permitteren: iemand is wat hij of zij zich kan aanschappen'' (OT, p. 64)
        \end{citaat}
        \begin{citaat}{Paul Verhaeghe over identiteit (p. 2)}
            Een tweede evolutionair bepaald kenmerk sluit daar onmiddelijk bij aan: wij zijneen hi\"erarchische soort, een groep bestaat nooit uit gelijke individuen, maar bevat altijd een sociale stratificatie.
        \end{citaat}
    \end{stelling-antwoord}
\end{examenvraag}


\begin{examenvraag}
    \begin{stelling}
        Individualisering betekent in Paul Verhaeghes termen: meer mogelijkheid tot separatie.
    \end{stelling}

    \begin{stelling-antwoord}{Waar}
        Separatie was moeilijk in de tijd waarin tradities absolute waarheden beschreven.
        Nu, met individualisering, deze tradities gerelativeerd worden, wordt het eenvoudig om zichzelf af te zetten tegen de spiegel die ons voorgehouden wordt.
        Separatie is hier het zich afzetten tegen de immer zwakker wordende dominante modellen.
        Separatie wordt makkelijk gemaakt door pluralisering, wat hand in hand gaat met individualisering.
        \begin{citaat}{Paul Verhaeghe over identiteit (p. 5)}
            Separatie betekent altijd een keuzer voor een andere indentificatie dan de dominante.
        \end{citaat}
        \begin{citaat}{Slides: Wat kenmerkt onze maatschapij? (19)}
            Individualisering gaat hand in hand met de pluralisering:
        \end{citaat}
        \begin{citaat}{Onderbroken traditie (p. 48)}
            Radicale pluraliteit is de basiskarakteristiek van onze tijd.
            Fundamenteel voor de postmoderniteit is de basiservaring dat een zelfde gegeven met evenveel recht vanuit volledig onderscheiden perspectieven kan beschouwd worden.
        \end{citaat}
    \end{stelling-antwoord}
\end{examenvraag}


\begin{examenvraag}
    \begin{stelling}
        Volgens Paul Verhaeghe leidt meer vrije markt tot meer vrijheid.
    \end{stelling}

    \begin{stelling-antwoord}{Onwaar}
        Paul Verhaeghe argumenteert dat een vrije markt tot minder individuele vrijheid leidt via de analogie met een callcenter.
        \begin{citaat}{Paul Verhaeghe over het neoliberalisme (p. 3)}
            Zowel het idee van maakbaarheid als eigen verantwoordelijkheid hebben op deze manier zowel een zeer enge als een zeer dwingende invulling gekregen.
            Dit kadert bij het belangrijkste en vermoedelijk ook het gevaarlijkste neoliberale idee: dat de huidige invulling van de vrije markt ons maximale individuele vrijheid zou opleveren en een minimum aan dwang.
            De facto is het omgekeerde meer en meer waar: de zogenaamde 'vrije' markt dwingt ons in een keurslijf met heel beperkte opties.
            Ik maak graag de vergelijking met een call center, waar een zoetgevooisde stem ons haar keuzepalet oplegt: 'Voor optie x, kies 1, voor optie y, kies 2, voor optie z, kies drie'.
            Ik beschouw dit als een metafoor voor het neoliberale bestel: het aantal opties waartussen we kunnen kiezen, is niet alleen erg beperkt, bovendien worden ze letterlijk gedicteerd.
        \end{citaat}
    \end{stelling-antwoord}
\end{examenvraag}


\begin{examenvraag}
    \begin{stelling}
        Volgens Paul Verhaeghe gaan meer vrije markt en meer democratie hand in hand.
    \end{stelling}

    \begin{stelling-antwoord}{Onwaar}
        Paul Verhaeghe geeft deze stelling als \'e\'en van de twee misvattingen over de vrije markt.
        Hij verwijst hiervoor naar een boek van een jonge Leuvense filosoof: Thomas Decreus.
        \begin{citaat}{Paul Verhaeghe over het neoliberalisme (p. 5)}
            De eerste, en meteen belangrijkste misvatting is dat vrije markt synoniem is met democratie, en dat meer vrije markt meer democratie betekent.
            Deze opvatting dateert uit de periode van de koude oorlog, toen we met recht en reden een tegenstelling konden zien tussen een centrale planeconomie en totalitaire regimes enerzijds en de toenmalige vrije markt en democratie anderzijds.
            Als opvatting verschijnt dit in de titels van toenmalige verdedigers van de vrije markt: Capitalism and Freedom van Milton Friedman (1962) en The road to serfdom van Friedrich von Hayek (1944).
            Wat toen juist was, geldt niet meer voor onze geglobaliseerde en gedigitaliseerde economie, en toch blijft dit geloof overeind.
            Nog steeds wordt de vrije markt verdedigd als noodzakelijke basis voor een democratie, terwijl het vandaag exact het omgekeerde is.
            Hoe meer vrije markt, hoe minder democratie, dat is de hoofdstelling van een jonge Leuvense filosoof, Thomas Decreus (2013), in een boek dat ik iedereen kan aanbevelen.
            De oorzaak daarvan ligt in het feit dat de overheid het idee van de vermarkting zo
        \end{citaat}
    \end{stelling-antwoord}
\end{examenvraag}


\begin{examenvraag}
    \begin{stelling}
        Empathie vraagt om sterk ontwikkelde cognitieve vermogens en komt bijgevolg slechts bij weinig dieren voor.
    \end{stelling}

    \begin{antwoord}
    \end{antwoord}
\end{examenvraag}


\begin{examenvraag}
    \begin{stelling}
        Gerichte hulp is wijdverbreid in het dierenrijk en moet bijgevolg reeds vroeg in de evolutie van het leven ontstaan zijn.
    \end{stelling}

    \begin{antwoord}
    \end{antwoord}
\end{examenvraag}


\begin{examenvraag}
    \begin{stelling}
        Volgens de cursus is een gezonde dosis egocentrisme noodzakelijk.
    \end{stelling}

	 \begin{stelling-antwoord}{Waar}
        Een dosis egocentrisme is nodig om aan de "verantwoordelijkheid in eerste persoon" te voldoen. 
        Je moet zelf werk maken van een zinvol en gelukkig bestaan. 
        Je moet je eigen weg smeden, en dit gaat niet als je nooit voor je eigen zelf kiest.
    \end{stelling-antwoord}
\end{examenvraag}


\begin{examenvraag}
    \begin{stelling}
        Het feit dat de priester en de leviet in de parabel van de Barmhartige Samaritaan met een wijde boog om de gewonde man heen lopen maakt duidelijk dat ze hem niet hebben zien liggen.
    \end{stelling}

	 \begin{stelling-antwoord}{Onwaar}
	   Zie blz 82 tekst 17.
	   
	   Ze zien de man wel degelijk. 
	   Het is ook geen "neutraal" zien.
	   Het feit dat ze omlopen symboliseert dat het niet prettig is voor de personen om de man te zien lijden.
	   Ze willen het in feite niet zien, omdat het lijden van de man hun in hun "rust" verstoord. 
	   Vanuit hun eigen reisplan/levensproject laten ze zich liever niet in met hem. 
	   (teveel verantwoordelijkheid in eerste persoon?)
	 
    \end{stelling-antwoord}
\end{examenvraag}


\begin{examenvraag}
    \begin{stelling}
        De verschijning van de lijdende ander leidt tot het einde van mijn vrijheid.
    \end{stelling}

	 \begin{stelling-antwoord}{Onwaar}
		Het onderbreekt je eigen levenspad, wat weerstand oproept.
		Ik behoud echter wel de "vrijheid van antwoord". (Tegenover van geen "vrijheid van iniatief")
		Ik ben vrij de verantwoordelijkheid niet op te nemen, de ander kan niets forceren.
		(Verantwoordelijkheid in tweede persoon - barmhartigheid)
		
	 \begin{citaat}{slides RZL04 - 16}
Tegelijk kan de ander niets forceren, niets afdwingen; de ander kan slechts oproepen, appelleren, smeken.
Dit is waarom mijn vrijheid behouden blijft: ik heb dan wel geen “vrijheid van initiatief” (de ander overvalt mij), 
maar ik behoud mijn “vrijheid van antwoord”.
De verantwoordelijkheid in de tweede persoon wordt mij aangedaan (door de ander die verschijnt).
En ik moet antwoorden (ik moet een keuze maken), maar ik ben wel vrij om die verantwoordelijkheid al dan niet op te nemen (ik kan ze ook afwijzen – ook al betekent dat een keuze voor het kwade en tegen het goede).
        \end{citaat}
		
    \end{stelling-antwoord}
\end{examenvraag}


\begin{examenvraag}
    \begin{stelling}
        Naastenliefde betekent dat je iedereen even sympathiek moet vinden.
    \end{stelling}

    \begin{stelling-antwoord}{Onwaar}
        Naastenliefde betekent dat je moet inzitten met het lijden van de medemens.
        Meer bepaald dat je je medemens niet alleen mag laten in zijn lijden.
        Dit is iets anders dan je medemens sympathiek te vinden.
        Het is geen vorm van "eros".
        Het dus geen vorm van verlangen, of een vorm van vriendschap.
    \end{stelling-antwoord}
\end{examenvraag}


\begin{examenvraag}
    \begin{stelling}
        Volgens de cursus is het absurd om van de naastenliefde een gebod te maken.
    \end{stelling}

    \TODO{kan iemand bevestigen dat ik hier het juist begrepen heb?}
    \begin{stelling-antwoord}{Onwaar}
        Zie p85-86 barmhartige samaritaan.
        Naastenliefde heeft niets met subjectieve selectie en uitverkiezing te maken.
        Ze vertrekt vanuit een objectieve gegevenheid dat de andere zich onaangemeld aan mij voordoet.
        Concreet betekent dit dat er in principe weinig baat voor uw bestaan bij is om een lijdende (vreemde) andere te helpen. 
        Maar de ander heeft recht op hulp, op liefde. 
        Vandaar ook het gebod: help de andere niet om dat jij dat wilt op basis van subjectiviteit, maar omdat die er recht op heeft.
        In deze zin is de naastenliefde de grondslag voor de mensenrechten.
    \end{stelling-antwoord}
\end{examenvraag}


\begin{examenvraag}
    \begin{stelling}
        De naastenliefde is een gebod omdat ze tegennatuurlijk is.
    \end{stelling}

    \TODO{kan iemand bevestigen dat ik hier het juist begrepen heb? aanvulling mss ook?}

    \begin{stelling-antwoord}{Waar}
        (Uitgelegd adh van termen barmhartige samaritaan.)
        Als mensen een 'onderbreking' van hun 'zijn' ondervinden, zijn ze geneigd om hun 'blik'
        af te wenden. 
        Zoals de priester en de leviet.
        Ze voelen een soort van weerstand om in te gaan op die onderbreking.
        Het dwarsboompt hun plan, hun levenspad.
        Net daarom is het nodig om hier een gebod van te maken.
    \end{stelling-antwoord}
\end{examenvraag}


\begin{examenvraag}
    \begin{stelling}
        De parabel van de Barmhartige Samaritaan leert ons alles over de ethische verhouding tussen mensen.
    \end{stelling}

    \begin{antwoord}
    \end{antwoord}
\end{examenvraag}


\begin{examenvraag}
    \begin{stelling}
        De parabel van de Barmhartige Samaritaan presenteert de Samaritaan waar de parabel naar genoemd is als een model van totale barmhartigheid.
    \end{stelling}

    \begin{stelling-antwoord}{Onwaar}
        Totale barmhartigheid zou betekenen dat de Samaritaan de man helemaal zelf zou verzorgd hebben.
        Tot in het oneindige.
        De Samaritaan echter, helpt de man tot in een herberg, en delegeert de zorg dan naar de waard.
        De Samaritaan zijn primaire doelen verdwijnen niet.
        Zijn verantwoordelijkheid in eerste persoon verdwijnt niet.
    \end{stelling-antwoord}
\end{examenvraag}


\begin{examenvraag}
    \begin{stelling}
        Gelovige wetenschappers zijn volgens het koofmodel eigenlijk schizofreen want je kan niet tegelijk een goede wetenschapper en een goede gelovige zijn.
    \end{stelling}

    \begin{antwoord}
    \end{antwoord}
\end{examenvraag}


\begin{examenvraag}
    \begin{stelling}
        Wie vandaag een harmonie tussen geloof en natuurwetenschap nastreeft, bevordert volgens Taede Smedes eigenlijk het conflict tussen beide.
    \end{stelling}

    \begin{antwoord}
    \end{antwoord}
\end{examenvraag}


\begin{examenvraag}
    \begin{stelling}
        Volgens het differentiemodel moeten gelovigen rekening houden met de resultaten van wetenschappelijk onderzoek.
    \end{stelling}

    \begin{antwoord}
    \end{antwoord}
\end{examenvraag}


\begin{examenvraag}
    \begin{stelling}
        Volgens het differentiemodel zijn de resultaten van wetenschappelijk onderzoek irrelevant voor het geloof.
    \end{stelling}

    \begin{antwoord}
    \end{antwoord}
\end{examenvraag}


\begin{examenvraag}
    \begin{stelling}
        Taede Smedes is een aanhanger van het kloofmodel want hij verdedigt een complete “boedelscheiding” tussen beide.
    \end{stelling}

    \begin{antwoord}
    \end{antwoord}
\end{examenvraag}


\end{document}
