\documentclass[main.tex]{subfiles}

\begin{document}
\section{Kernbegrippen en sleutelfiguren}


\begin{examenvraag}
    \begin{vraag}
        Identificatie
    \end{vraag}

    \begin{antwoord}
        Identificatie, in de cursus, heeft betrekking tot...
    \end{antwoord}
\end{examenvraag}


\begin{examenvraag}
    \begin{vraag}
        Separatie
    \end{vraag}

    \begin{antwoord}

    \end{antwoord}
\end{examenvraag}


\begin{examenvraag}
    \begin{vraag}
        Eudaimonia
    \end{vraag}

    \begin{antwoord}

    \end{antwoord}
\end{examenvraag}


\begin{examenvraag}
    \begin{vraag}
        Detraditionalisering
    \end{vraag}

    \begin{antwoord}

    \end{antwoord}
\end{examenvraag}


\begin{examenvraag}
    \begin{vraag}
        Keuzebiografie
    \end{vraag}

    \begin{antwoord}

    \end{antwoord}
\end{examenvraag}


\begin{examenvraag}
    \begin{vraag}
        Radicale pluraliteit
    \end{vraag}

    \begin{antwoord}

    \end{antwoord}
\end{examenvraag}


\begin{examenvraag}
    \begin{vraag}
        Het ideaal van de romantische liefde
    \end{vraag}

    \begin{antwoord}

    \end{antwoord}
\end{examenvraag}


\begin{examenvraag}
    \begin{vraag}
        Wegwerpcultuur
    \end{vraag}

    \begin{antwoord}

    \end{antwoord}
\end{examenvraag}


\begin{examenvraag}
    \begin{vraag}
        Hedonisme
    \end{vraag}

    \begin{antwoord}

    \end{antwoord}
\end{examenvraag}


\begin{examenvraag}
    \begin{vraag}
        Ervaringshonger
    \end{vraag}

    \begin{antwoord}

    \end{antwoord}
\end{examenvraag}


\begin{examenvraag}
    \begin{vraag}
        Kick
    \end{vraag}

    \begin{antwoord}

    \end{antwoord}
\end{examenvraag}


\begin{examenvraag}
    \begin{vraag}
        Lichaamscultuur
    \end{vraag}

    \begin{antwoord}

    \end{antwoord}
\end{examenvraag}


\begin{examenvraag}
    \begin{vraag}
        Hypecultuur
    \end{vraag}

    \begin{antwoord}

    \end{antwoord}
\end{examenvraag}


\begin{examenvraag}
    \begin{vraag}
        Mediatisering
    \end{vraag}

    \begin{antwoord}

    \end{antwoord}
\end{examenvraag}


\begin{examenvraag}
    \begin{vraag}
        Economisering
    \end{vraag}

    \begin{antwoord}

    \end{antwoord}
\end{examenvraag}


\begin{examenvraag}
    \begin{vraag}
        Vermarkting
    \end{vraag}

    \begin{antwoord}

    \end{antwoord}
\end{examenvraag}


\begin{examenvraag}
    \begin{vraag}
        Rank and Yank
    \end{vraag}

    \begin{antwoord}

    \end{antwoord}
\end{examenvraag}


\begin{examenvraag}
    \begin{vraag}
        Loonspanning
    \end{vraag}

    \begin{antwoord}

    \end{antwoord}
\end{examenvraag}


\begin{examenvraag}
    \begin{vraag}
        Crowding out
    \end{vraag}

    \begin{antwoord}

    \end{antwoord}
\end{examenvraag}


\begin{examenvraag}
    \begin{vraag}
        Moraal
    \end{vraag}

    \begin{antwoord}

    \end{antwoord}
\end{examenvraag}


\begin{examenvraag}
    \begin{vraag}
        Descriptieve ethiek
    \end{vraag}

    \begin{antwoord}

    \end{antwoord}
\end{examenvraag}


\begin{examenvraag}
    \begin{vraag}
        Prescriptieve ethiek
    \end{vraag}

    \begin{antwoord}

    \end{antwoord}
\end{examenvraag}


\begin{examenvraag}
    \begin{vraag}
        Normatieve ethiek
    \end{vraag}

    \begin{antwoord}

    \end{antwoord}
\end{examenvraag}


\begin{examenvraag}
    \begin{vraag}
        Empathie
    \end{vraag}

    \begin{antwoord}

    \end{antwoord}
\end{examenvraag}


\begin{examenvraag}
    \begin{vraag}
        De zuilen van de moraliteit
    \end{vraag}

    \begin{antwoord}

    \end{antwoord}
\end{examenvraag}


\begin{examenvraag}
    \begin{vraag}
        De Baal Sjem Tov
    \end{vraag}

    \begin{antwoord}

    \end{antwoord}
\end{examenvraag}


\begin{examenvraag}
    \begin{vraag}
        Dubbelgebod van de liefde
    \end{vraag}

    \begin{antwoord}

    \end{antwoord}
\end{examenvraag}


\begin{examenvraag}
    \begin{vraag}
        Verantwoordelijkheid in de eerste
    \end{vraag}

    \begin{antwoord}

    \end{antwoord}
\end{examenvraag}


\begin{examenvraag}
    \begin{vraag}
        persoon
    \end{vraag}

    \begin{antwoord}

    \end{antwoord}
\end{examenvraag}


\begin{examenvraag}
    \begin{vraag}
        Verantwoordelijkheid in de tweede
    \end{vraag}

    \begin{antwoord}

    \end{antwoord}
\end{examenvraag}


\begin{examenvraag}
    \begin{vraag}
        persoon
    \end{vraag}

    \begin{antwoord}

    \end{antwoord}
\end{examenvraag}


\begin{examenvraag}
    \begin{vraag}
        Heteronomie
    \end{vraag}

    \begin{antwoord}

    \end{antwoord}
\end{examenvraag}


\begin{examenvraag}
    \begin{vraag}
        Barmhartigheid
    \end{vraag}

    \begin{antwoord}

    \end{antwoord}
\end{examenvraag}


\begin{examenvraag}
    \begin{vraag}
        Verantwoordelijkheid in de derde
    \end{vraag}

    \begin{antwoord}

    \end{antwoord}
\end{examenvraag}


\begin{examenvraag}
    \begin{vraag}
        persoon
    \end{vraag}

    \begin{antwoord}

    \end{antwoord}
\end{examenvraag}


\begin{examenvraag}
    \begin{vraag}
        Alteriteitservaring
    \end{vraag}

    \begin{antwoord}

    \end{antwoord}
\end{examenvraag}


\begin{examenvraag}
    \begin{vraag}
        De “list” waarvan sprake in Schindler’s List
    \end{vraag}

    \begin{antwoord}

    \end{antwoord}
\end{examenvraag}


\begin{examenvraag}
    \begin{vraag}
        Kloofmodel
    \end{vraag}

    \begin{antwoord}

    \end{antwoord}
\end{examenvraag}


\begin{examenvraag}
    \begin{vraag}
        Het NOMA-principe
    \end{vraag}

    \begin{antwoord}

    \end{antwoord}
\end{examenvraag}


\begin{examenvraag}
    \begin{vraag}
        Stephen Jay Gould
    \end{vraag}

    \begin{antwoord}

    \end{antwoord}
\end{examenvraag}


\begin{examenvraag}
    \begin{vraag}
        Harmoniemodel
    \end{vraag}

    \begin{antwoord}

    \end{antwoord}
\end{examenvraag}


\begin{examenvraag}
    \begin{vraag}
        Differentiemodel
    \end{vraag}

    \begin{antwoord}

    \end{antwoord}
\end{examenvraag}



\end{document}
