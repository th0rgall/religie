\documentclass[main.tex]{subfiles}

\begin{document}
\section{Kernbegrippen en sleutelfiguren: ``Wat is...? Geef telkens een omschrijving.''}


\begin{examenvraag}
    \begin{vraag}
        Identificatie
    \end{vraag}

    \begin{antwoord}
        Identificatie is het proces waarin we `onszelf' worden door ons te spiegelen aan onze omgeving. Onze omgeving zegt ons wie we zijn, moeten zijn en niet mogen zijn.
        \begin{citaat}{Paul Verhaeghe over identiteit}
            Identificatie heeft dezelfde etymologische grond als identiteit, met name het Latijnse idem, gelijk.
            Wij worden 'onszelf', tussen aanhalingstekens, door zoveel mogelijk te gelijken op de spiegel die ons voorgehouden wordt.
            De moderne wetenschappelijke benaming voor identiteit is dan ook spiegeling of `mirroring'.
        \end{citaat}
    \end{antwoord}
\end{examenvraag}


\begin{examenvraag}
    \begin{vraag}
        Separatie
    \end{vraag}

    \begin{antwoord}
        Separatie is het proces waarin we `onszelf' worden door ons af te zetten tegen onze omgeving. De mogelijkheid voor andere invullingen van onze indentiteit, gebaseerd op andere verhalen.
        \begin{citaat}{Paul Verhaeghe over identiteit}                                     
            Het separatieproces en het bijbehorend streven naar autonomie zijn even belangrijk voor onze identiteit als de identificatie, omdat we daarmee een eigenheid ontwikkelen door ons af te zetten en actief een keuze te maken.                                                     
        \end{citaat} 
    \end{antwoord}
\end{examenvraag}


\begin{examenvraag}
    \begin{vraag}
        Eudaimonia
    \end{vraag}

    \begin{antwoord}
        In de cursus wordt een mogelijk levensdoel beschreven waarin het persoonlijk welzijn, 'gelukkig' zijn, beschreven wordt als het hoogste goed.
        Dit is in de lijn van wat Aristoteles `Eudaimonia` noemde.
        Letterlijk vertaald betekent het `het menselijk openbloeien'.
        \begin{citaat}{Wikipedia: ``Eudaimonia''}
            Eudaimonia, sometimes anglicized as eudaemonia or eudemonia, is a Greek word commonly translated as happiness or welfare; however, "human flourishing" has been proposed as a more accurate translation.
            [...]
            In Aristotle's works, eudaimonia was (based on older Greek tradition) used as the term for the highest human good, and so it is the aim of practical philosophy, including ethics and political philosophy, to consider (and also experience) what it really is, and how it can be achieved.
        \end{citaat}
        \begin{citaat}{Borderline times: Over het hedonisme (de genotscultuur) (p. 67)}
            Genieten is in dit verhaal niet enkel de nieuwe moraal geworden, het is ook ieders plicht.
            `Gij zult genieten' heeft nagenoeg alle andere geboden van de troon gestoten.
        \end{citaat}
    \end{antwoord}
\end{examenvraag}


\begin{examenvraag}
    \begin{vraag}
        Detraditionalisering
    \end{vraag}

    \begin{antwoord}
        \'e\'en van de twee delen van individualisering, naast `emancipatie'.
        Detraditionalisering is de relativering van zowel de inhoud van traditie als van de traditieoverlevering.

        \begin{citaat}{Slides: ``Wat kenmerkt onze maatschappij (15)''}
            Detraditionalisering = “de relativering van alle traditionele normen en waarden”; “collectief gedragen opvattingen”, alledaagse handelingen en omgangsvormen verliezen hun dwingend karakter.
        \end{citaat}
        \begin{citaat}{In de greep van ``de moderne tijd''}
            Vooreerst behelst het een ingrijpende `de-traditionalisering' (Beck) van de toegenomen individuele beslissingsvrijheid inzake waarden, normen en alledaagse handelingen (kleding, vrije tijd, voeding,...).
        \end{citaat}
    \end{antwoord}
\end{examenvraag}

\begin{examenvraag}
    % Extra
    \begin{vraag}
        Traditie
    \end{vraag}

    \begin{antwoord}
    \end{antwoord}
\end{examenvraag}

\begin{examenvraag}
    % Extra
    \begin{vraag}
        Bricolage
    \end{vraag}

    \begin{antwoord}
    \end{antwoord}
\end{examenvraag}

\begin{examenvraag}
    % Extra
    \begin{vraag}
        Emancipatie
    \end{vraag}

    \begin{antwoord}
    \end{antwoord}
\end{examenvraag}


\begin{examenvraag}
    \begin{vraag}
        Keuzebiografie
    \end{vraag}

    \begin{antwoord}
        Het kunnen kiezen wat iemand doet met diens leven.
        In de cursus wordt het niet alleen positief bekeken.
        De last van de keuze veroorzaakt ook stress bij de mensen van deze generatie en jaloezie bij de mensen van een oudere generatie.
        Het kunnen kiezen brengt ook met zich mee dat we moeten kiezen.
        \begin{citaat}{Slides: ``Wat kenmerkt onze maatschappij (17)''}
            \begin{itemize}
                \item Kunnen kiezen = moeten kiezen
                \item Identiteitsconstructie wordt een opdracht en zelfs een plicht we worden gedwongen de managers van ons leven te worden, het planningsbureau van onze levensloop
                \item Impliciete boodschap: `Je kan het maken, dus je moet het maken'' en wie het niet maakt, neemt zijn/haar verantwoordelijkheid niet op en is schuldig (profiteert)
            \end{itemize}
        \end{citaat}
        \begin{citaat}{In de greep van ``de moderne tijd''}
            Met de ovegang ``van een normaal- naar een keuze-biografie'' (H. Ley) veranderde het alledaagse leven van een toenemend aantal mensen vanaf het einde van de jaren vijftig in een schier eindeloze beslissingsketen.
            `Ge\"individiualiseerde individueen' kunnen niet enkel zelf keuzes maken, ze moeten dat vaak ook doen.
        \end{citaat}
        \begin{citaat}{Borderline Times: Over individualisering als last en zegen (p. 100)}
            De boodschap is dat we het zelf maar moeten uitzoeken, dat we het zelf mogen kiezen.
            Dat de `drang' om open te bloeien dan ook nog eens gelinkt is aan succes in wat we ondernemen, legt de lat nog hoger.
        \end{citaat}
        \begin{citaat}{Het liefste deed ik elk halfjaar iets anders}
            Alleen is generatie Y opgegroeid met meer luxe en kansen, zoals hogere studies, waardoor de verwachtingen hoger gespannen staan.
            `Uitblinken' en uniek zijn is de boodschap voor deze generatie.
            [...]
            Piano of saxifoon, voetbal of ballet, hogeschool of universiteit, met vlees of zonder? `Ja, we zijn opgegroeid met veel vanzelfsprekendheden en we moeten ons eten niet eerst uit het veld trekken.' geeft Britt Verhelst toe.
            `Dankzij die vrijheid kunnen we op een andere manier over het leven nadenken.
            Maar daardoor kan ik niet gewoon beslissen: `ok\'e dat ga ik doen. Het liefste deed ik elk halfjaar iets anders.'
        \end{citaat}
    \end{antwoord}
\end{examenvraag}


\begin{examenvraag}
    \begin{vraag}
        Radicale pluraliteit
    \end{vraag}

    \begin{antwoord}
        Radicale pluraliteit is de notie die beschrijft dat er geen universeel eenheidsperspectief meer bestaat.
        Alles moet vanuit vanuit verschillende perspectieven kunnen beschouwd worden, ookal zijn die perspectieven niet altijd even ondeling compatibel.
        \begin{citaat}{Slides: ``Wat kenmerkt onze maatschappij (20)''}
            Lieven Boeve maakt van “radicale pluraliteit” zelfs “de basiskarakteristiek van onze tijd”.
        \end{citaat}
        \begin{citaat}{Onderbroken traditie (p. 48)}
            Fundamenteel voor de postmoderniteit is de basiservaring dat een zelfde gegeven met evenveel recht vanuit volledig onderscheiden perspectieven kan beschouwd worden.
            Hierbij heeft elk gezichtspunt waarde op zich, ook al tonen ze zich onderling weinig compatibel, en vaak zelfs conflictueus.
            Het universeel eenheidsperspectief is niet meer; de grote verhalen hebben afgedaan.
            Deze basiservaring van de postmoderne tijd doet zich voor in heel diverse gebieden van de menselijke leefwereld, zowel in literatuur, architectuur, sculptuur en schilderkunst, als in cultuur- en wetenschapsfilosofie, economie en politiek.
            In elk van deze gebieden duikt een veelheid aan vaak niet met elkaar te verzoenen perspectieven, taalspelen, werk- en denkwijzen op.
            Deze pluraliteit zet zich door als vooronderstelling van alle denken en handelen, wat onmiddellijk impliceert dat niemand nog zomaar kan beweren de waarheid in pacht te hebben.
            Het postmoderne kritische bewustzijn verzet zich principieel tegen elke opgeëiste hegemonie.
            Elke universalistische pretentie wordt kritisch ontmaskerd als een verabsoluteerd particulier gezichtspunt.
        \end{citaat}
    \end{antwoord}
\end{examenvraag}


\begin{examenvraag}
    \begin{vraag}
        Het ideaal van de romantische liefde
    \end{vraag}

    \begin{antwoord}

    \end{antwoord}
\end{examenvraag}


\begin{examenvraag}
    \begin{vraag}
        Wegwerpcultuur
    \end{vraag}

    \begin{antwoord}

    \end{antwoord}
\end{examenvraag}


\begin{examenvraag}
    \begin{vraag}
        Hedonisme
    \end{vraag}

    \begin{antwoord}

    \end{antwoord}
\end{examenvraag}


\begin{examenvraag}
    \begin{vraag}
        Ervaringshonger
    \end{vraag}

    \begin{antwoord}

    \end{antwoord}
\end{examenvraag}


\begin{examenvraag}
    \begin{vraag}
        Kick
    \end{vraag}

    \begin{antwoord}

    \end{antwoord}
\end{examenvraag}


\begin{examenvraag}
    \begin{vraag}
        Lichaamscultuur
    \end{vraag}

    \begin{antwoord}

    \end{antwoord}
\end{examenvraag}


\begin{examenvraag}
    \begin{vraag}
        Hypecultuur
    \end{vraag}

    \begin{antwoord}

    \end{antwoord}
\end{examenvraag}


\begin{examenvraag}
    \begin{vraag}
        Mediatisering
    \end{vraag}

    \begin{antwoord}

    \end{antwoord}
\end{examenvraag}


\begin{examenvraag}
    \begin{vraag}
        Economisering
    \end{vraag}

    \begin{antwoord}

    \end{antwoord}
\end{examenvraag}


\begin{examenvraag}
    \begin{vraag}
        Vermarkting
    \end{vraag}

    \begin{antwoord}

    \end{antwoord}
\end{examenvraag}


\begin{examenvraag}
    \begin{vraag}
        Rank and Yank
    \end{vraag}

    \begin{antwoord}

    \end{antwoord}
\end{examenvraag}


\begin{examenvraag}
    \begin{vraag}
        Loonspanning
    \end{vraag}

    \begin{antwoord}

    \end{antwoord}
\end{examenvraag}


\begin{examenvraag}
    \begin{vraag}
        Crowding out
    \end{vraag}

    \begin{antwoord}

    \end{antwoord}
\end{examenvraag}


\begin{examenvraag}
    \begin{vraag}
        Moraal
    \end{vraag}

    \begin{antwoord}

    \end{antwoord}
\end{examenvraag}


\begin{examenvraag}
    \begin{vraag}
        Descriptieve ethiek
    \end{vraag}

    \begin{antwoord}

    \end{antwoord}
\end{examenvraag}


\begin{examenvraag}
    \begin{vraag}
        Prescriptieve ethiek
    \end{vraag}

    \begin{antwoord}

    \end{antwoord}
\end{examenvraag}


\begin{examenvraag}
    \begin{vraag}
        Normatieve ethiek
    \end{vraag}

    \begin{antwoord}

    \end{antwoord}
\end{examenvraag}


\begin{examenvraag}
    \begin{vraag}
        Empathie
    \end{vraag}

    \begin{antwoord}

    \end{antwoord}
\end{examenvraag}


\begin{examenvraag}
    \begin{vraag}
        De zuilen van de moraliteit
    \end{vraag}

    \begin{antwoord}

    \end{antwoord}
\end{examenvraag}


\begin{examenvraag}
    \begin{vraag}
        De Baal Sjem Tov
    \end{vraag}

    \begin{antwoord}

    \end{antwoord}
\end{examenvraag}


\begin{examenvraag}
    \begin{vraag}
        Dubbelgebod van de liefde
    \end{vraag}

    \begin{antwoord}

    \end{antwoord}
\end{examenvraag}


\begin{examenvraag}
    \begin{vraag}
        Verantwoordelijkheid in de eerste persoon
    \end{vraag}

    \begin{antwoord}

    \end{antwoord}
\end{examenvraag}


\begin{examenvraag}
    \begin{vraag}
        Verantwoordelijkheid in de tweede persoon
    \end{vraag}

    \begin{antwoord}

    \end{antwoord}
\end{examenvraag}


\begin{examenvraag}
    \begin{vraag}
        Heteronomie
    \end{vraag}

    \begin{antwoord}

    \end{antwoord}
\end{examenvraag}


\begin{examenvraag}
    \begin{vraag}
        Barmhartigheid
    \end{vraag}

    \begin{antwoord}

    \end{antwoord}
\end{examenvraag}


\begin{examenvraag}
    \begin{vraag}
        Verantwoordelijkheid in de derde persoon
    \end{vraag}

    \begin{antwoord}

    \end{antwoord}
\end{examenvraag}


\begin{examenvraag}
    \begin{vraag}
        Alteriteitservaring
    \end{vraag}

    \begin{antwoord}

    \end{antwoord}
\end{examenvraag}


\begin{examenvraag}
    \begin{vraag}
        De “list” waarvan sprake in Schindler’s List
    \end{vraag}

    \begin{antwoord}
        Oskar Shindler heeft een duizendtal gevangenen, vooral joden, geredt uit het concentratiekamp van Plaseow door ze te werk te stellen in zijn fabriek in Krakow.
        Hij kocht de joden van de kampcommandant.
        Hij bezorgde de commandant een lijst met fictieve jobs voor de joden die hij kocht om de commandant te overtuigen dat de joden van vitaal belang waren, als werkkracht, voor de oorlog.
        Hij redde zo 1200 joden van de dood in het concentratiekamp.
        "The list is absolute good. The list is life, around the edges is a gulf", wat zoveel betekent als "Diegenen die niet op de lijst stonden overleefden niet.", is hieromtrent een markant citaat.
    \end{antwoord}
\end{examenvraag}


\begin{examenvraag}
    \begin{vraag}
        Kloofmodel
    \end{vraag}

    \begin{antwoord}

    \end{antwoord}
\end{examenvraag}


\begin{examenvraag}
    \begin{vraag}
        Het NOMA-principe
    \end{vraag}

    \begin{antwoord}

    \end{antwoord}
\end{examenvraag}


\begin{examenvraag}
    \begin{vraag}
        Stephen Jay Gould
    \end{vraag}

    \begin{antwoord}

    \end{antwoord}
\end{examenvraag}


\begin{examenvraag}
    \begin{vraag}
        Harmoniemodel
    \end{vraag}

    \begin{antwoord}

    \end{antwoord}
\end{examenvraag}


\begin{examenvraag}
    \begin{vraag}
        Differentiemodel
    \end{vraag}

    \begin{antwoord}

    \end{antwoord}
\end{examenvraag}



\end{document}
