\documentclass[main.tex]{subfiles}

\begin{document}
\section{Kernbegrippen en sleutelfiguren: ``Wat is...? Geef telkens een omschrijving.''}


\begin{examenvraag}
    \begin{vraag}
        Identificatie
    \end{vraag}

    \begin{antwoord}
        Identificatie is het proces waarin we `onszelf' worden door ons te spiegelen aan onze omgeving. Onze omgeving zegt ons wie we zijn, moeten zijn en niet mogen zijn.
        \begin{citaat}{Paul Verhaeghe over identiteit}
            Identificatie heeft dezelfde etymologische grond als identiteit, met name het Latijnse idem, gelijk.
            Wij worden 'onszelf', tussen aanhalingstekens, door zoveel mogelijk te gelijken op de spiegel die ons voorgehouden wordt.
            De moderne wetenschappelijke benaming voor identiteit is dan ook spiegeling of `mirroring'.
        \end{citaat}
    \end{antwoord}
\end{examenvraag}


\begin{examenvraag}
    \begin{vraag}
        Separatie
    \end{vraag}

    \begin{antwoord}
        Separatie is het proces waarin we `onszelf' worden door ons af te zetten tegen onze omgeving. De mogelijkheid voor andere invullingen van onze indentiteit, gebaseerd op andere verhalen.
        \begin{citaat}{Paul Verhaeghe over identiteit}                                     
            Het separatieproces en het bijbehorend streven naar autonomie zijn even belangrijk voor onze identiteit als de identificatie, omdat we daarmee een eigenheid ontwikkelen door ons af te zetten en actief een keuze te maken.                                                     
        \end{citaat} 
    \end{antwoord}
\end{examenvraag}


\begin{examenvraag}
    \begin{vraag}
        Eudaimonia
    \end{vraag}

    \begin{antwoord}
        De invulling van geluk volgens Aristoteles.‭ ‬Geluk is de toestand van‭ "‬goed mens‭" ‬zijn,‭ ‬het verstand ten volle benutten want dit onderscheid ons van dieren.
        \begin{citaat}{Wikipedia: ``Eudaimonia''}
            Eudaimonia, sometimes anglicized as eudaemonia or eudemonia, is a Greek word commonly translated as happiness or welfare; however, "human flourishing" has been proposed as a more accurate translation.
            [...]
            In Aristotle's works, eudaimonia was (based on older Greek tradition) used as the term for the highest human good, and so it is the aim of practical philosophy, including ethics and political philosophy, to consider (and also experience) what it really is, and how it can be achieved.
        \end{citaat}
        \begin{citaat}{Borderline times: Over het hedonisme (de genotscultuur) (p. 67)}
            Genieten is in dit verhaal niet enkel de nieuwe moraal geworden, het is ook ieders plicht.
            `Gij zult genieten' heeft nagenoeg alle andere geboden van de troon gestoten.
        \end{citaat}
    \end{antwoord}
\end{examenvraag}


\begin{examenvraag}
    \begin{vraag}
        Detraditionalisering
    \end{vraag}

    \begin{antwoord}
        Een van de twee delen van individualisering, naast `emancipatie'.
        Detraditionalisering is de relativering van zowel de inhoud van traditie als van de traditieoverlevering.

        \begin{citaat}{Slides: ``Wat kenmerkt onze maatschappij (15)''}
            Detraditionalisering = “de relativering van alle traditionele normen en waarden”; “collectief gedragen opvattingen”, alledaagse handelingen en omgangsvormen verliezen hun dwingend karakter.
        \end{citaat}
        \begin{citaat}{In de greep van ``de moderne tijd''}
            Vooreerst behelst het een ingrijpende `de-traditionalisering' (Beck) van de toegenomen individuele beslissingsvrijheid inzake waarden, normen en alledaagse handelingen (kleding, vrije tijd, voeding,...).
        \end{citaat}
    \end{antwoord}
\end{examenvraag}

\begin{examenvraag}
    % Extra
    \begin{vraag}
        Traditie
    \end{vraag}

    \begin{antwoord}
    
    \begin{citaat}{Over traditie}
    Traditie is de overlevering, het gebruikelijke, het totaal van processen en 
    instituten, waardoor van geslacht tot geslcaht de verworven inzichten,
    bekwaamheden en instellingen worden overgelverd, en vervolgends ook het totaal 
    van het aldus overgeleverde.
    \end{citaat}
    \end{antwoord}
\end{examenvraag}

\begin{examenvraag}
    % Extra
    \begin{vraag}
        Bricolage
    \end{vraag}

    \begin{antwoord}
		Frans voor -do it yourself-. In de context van de cursus slaagt het op de ori\"{e}nteringsloze mens die zijn identiteit koopt. 
		  \begin{citaat}{Boek p63}
		De gehele levensstijl die men zich aanmeet, wordt aangekocht op de markt van vraag en aanbod.
        \end{citaat}
       \begin{citaat}{Boek p64}
		Identiteit is een kwestie van bricolage.
        \end{citaat}
    \end{antwoord}
   
\end{examenvraag}

\begin{examenvraag}
    % Extra
    \begin{vraag}
        Emancipatie
    \end{vraag}

    \begin{antwoord}
   		Een van de twee delen van individualisering, naast `detraditionalisering'. Emancipatie is het zelf actief kiezen van een mogelijkheid, een verhaal of zelfs meerdere verhalen.
   		\begin{citaat}{Slides: ``Wat kenmerkt onze maatschappij (15)''}
   		            Emancipatie, vrijmaking = er liggen voor elk individu in de wereld vele mogelijkheden open en het individu kan in principe zelf kiezen welke het wil benutten – dit uiteraard wel binnen bepaalde grenzen.
 		\end{citaat}
    \end{antwoord}
\end{examenvraag}


\begin{examenvraag}
    \begin{vraag}
        Keuzebiografie
    \end{vraag}

    \begin{antwoord}
        Het kunnen kiezen wat iemand doet met diens leven.
        In de cursus wordt het niet alleen positief bekeken.
        De last van de keuze veroorzaakt ook stress bij de mensen van deze generatie en jaloezie bij de mensen van een oudere generatie.
        Het kunnen kiezen brengt ook met zich mee dat we moeten kiezen.
        \begin{citaat}{Slides: ``Wat kenmerkt onze maatschappij (17)''}
            Individualisering = de overgang naar ``een keuze-biografie''.
        \end{citaat}
        \begin{citaat}{Slides: ``Wat kenmerkt onze maatschappij (17)''}
            \begin{itemize}
                \item Kunnen kiezen = moeten kiezen
                \item Identiteitsconstructie wordt een opdracht en zelfs een plicht we worden gedwongen de managers van ons leven te worden, het planningsbureau van onze levensloop
                \item Impliciete boodschap: `Je kan het maken, dus je moet het maken'' en wie het niet maakt, neemt zijn/haar verantwoordelijkheid niet op en is schuldig (profiteert)
            \end{itemize}
        \end{citaat}
        \begin{citaat}{In de greep van ``de moderne tijd''}
            Met de ovegang ``van een normaal- naar een keuze-biografie'' (H. Ley) veranderde het alledaagse leven van een toenemend aantal mensen vanaf het einde van de jaren vijftig in een schier eindeloze beslissingsketen.
            `Ge\"individiualiseerde individueen' kunnen niet enkel zelf keuzes maken, ze moeten dat vaak ook doen.
        \end{citaat}
        \begin{citaat}{Borderline Times: Over individualisering als last en zegen (p. 100)}
            De boodschap is dat we het zelf maar moeten uitzoeken, dat we het zelf mogen kiezen.
            Dat de `drang' om open te bloeien dan ook nog eens gelinkt is aan succes in wat we ondernemen, legt de lat nog hoger.
        \end{citaat}
        \begin{citaat}{Het liefste deed ik elk halfjaar iets anders}
            Alleen is generatie Y opgegroeid met meer luxe en kansen, zoals hogere studies, waardoor de verwachtingen hoger gespannen staan.
            `Uitblinken' en uniek zijn is de boodschap voor deze generatie.
            [...]
            Piano of saxifoon, voetbal of ballet, hogeschool of universiteit, met vlees of zonder? `Ja, we zijn opgegroeid met veel vanzelfsprekendheden en we moeten ons eten niet eerst uit het veld trekken.' geeft Britt Verhelst toe.
            `Dankzij die vrijheid kunnen we op een andere manier over het leven nadenken.
            Maar daardoor kan ik niet gewoon beslissen: `ok\'e dat ga ik doen. Het liefste deed ik elk halfjaar iets anders.'
        \end{citaat}
    \end{antwoord}
\end{examenvraag}


\begin{examenvraag}
    \begin{vraag}
        Radicale pluraliteit
    \end{vraag}

    \begin{antwoord}
        Radicale pluraliteit is de notie die beschrijft dat er geen universeel eenheidsperspectief meer bestaat.
        Alles moet vanuit vanuit verschillende perspectieven kunnen beschouwd worden, ookal zijn die perspectieven niet altijd even ondeling compatibel.
        \begin{citaat}{Slides: ``Wat kenmerkt onze maatschappij (20)''}
            Lieven Boeve maakt van “radicale pluraliteit” zelfs “de basiskarakteristiek van onze tijd”.
        \end{citaat}
        \begin{citaat}{Onderbroken traditie (p. 48)}
            Fundamenteel voor de postmoderniteit is de basiservaring dat een zelfde gegeven met evenveel recht vanuit volledig onderscheiden perspectieven kan beschouwd worden.
            Hierbij heeft elk gezichtspunt waarde op zich, ook al tonen ze zich onderling weinig compatibel, en vaak zelfs conflictueus.
            Het universeel eenheidsperspectief is niet meer; de grote verhalen hebben afgedaan.
            Deze basiservaring van de postmoderne tijd doet zich voor in heel diverse gebieden van de menselijke leefwereld, zowel in literatuur, architectuur, sculptuur en schilderkunst, als in cultuur- en wetenschapsfilosofie, economie en politiek.
            In elk van deze gebieden duikt een veelheid aan vaak niet met elkaar te verzoenen perspectieven, taalspelen, werk- en denkwijzen op.
            Deze pluraliteit zet zich door als vooronderstelling van alle denken en handelen, wat onmiddellijk impliceert dat niemand nog zomaar kan beweren de waarheid in pacht te hebben.
            Het postmoderne kritische bewustzijn verzet zich principieel tegen elke opgeëiste hegemonie.
            Elke universalistische pretentie wordt kritisch ontmaskerd als een verabsoluteerd particulier gezichtspunt.
        \end{citaat}
    \end{antwoord}
\end{examenvraag}


\begin{examenvraag}
    \begin{vraag}
        Het ideaal van de romantische liefde
    \end{vraag}

    \begin{antwoord}
        Liefde vanuit romantiek, tegenover liefde vanuit economische overwegingen is een ideaal dat nog maar een aantal decennia bestaat.
        Het idee dat we relaties kunnen invullen met `elkaar graag zien' noemen we het ideaal van de romantische liefde.
        \begin{citaat}{Borderline Times: Over het ideaal van de romantische liefde (p. 59)}
            In het begin van de 19de eeuw kwamen we in onze West-Europese cultuur op het idee dat relaties konden worden ingevuld met ‘elkaar graag zien’, en werd de romantische liefde een ideaal.
            Terwijl ze in oorsprong een zaak was van de burgerij, werd vlug ook de arbeidersklasse ervan doordrongen, en als laatste verschuiving: ze was binnengedrongen - én werd aanvaard - in de gelederen van de hogere adel.
            De grote verwachtingen die hierdoor binnen het kader van een relatie ontstonden, waren voorheen in het geheel niet aan de orde. Huwelijksregelingen werden afgesproken op het niveau van andere dan materiële overeenkomsten in de samenleving.
            In de 19de eeuw wordt ‘liefde’ het richtsnoer, de noodzakelijke basis, de voorwaarde, en ook nog het continue evaluatiecriterium van de relatie.
            Relaties krijgen een positieve invulling, zijn geen economische gegevenheden meer.
        \end{citaat}
    \end{antwoord}
\end{examenvraag}


\begin{examenvraag}
    \begin{vraag}
        Wegwerpcultuur
    \end{vraag}

    \begin{antwoord}
        We kunnen meer produceren dan we kunnen consumeren.
        Om dus te kunnen blijven consumeren moet er worden weggegooit; zelf als dat nog niet nodig was.
        De manier waarop deze economische tendens zich verder zet in onze cultuur noemt men de `wegwerpcultuur'.
        Een voorbeeld hiervan is hoe we al snel naar een nieuwe relatie zouden gaan zoeken als de onze slecht gaat.
        \begin{citaat}{Borderline Times: Over de impact van de consumptie-/wegwerpcultuur (p. 59)}
            Het is een feit dat de duur van alles daarmee ook korter en korter wordt.
            10 jaar dezelfde auto gebruiken was vroeger niets om je over te schamen.
            Nu heb je na 5 jaar al het gevoel met een stuk industriële archeologie rond te rijden.
            En als je dat gevoel zelf niet hebt, dan zorgt je omgeving er wel voor dat je het snel te pakken krijgt.
            [...]
            Het is intussen zover gekomen dat de wegwerpcultuur zich ook in ons relationele leven geïnstalleerd heeft, alsof wij tegenwoordig onze geliefde `kopen'.
            Zo zijn relaties `consumptiegoederen' geworden en dus, inwissel- en vervangbaar.
            In een relatie is bijgevolg enkel nog aandacht voor de `goede dagen'; de `kwade dagen' mogen er niet meer zijn, en toch heeft iedereen er.
        \end{citaat}
    \end{antwoord}
\end{examenvraag}


\begin{examenvraag}
    \begin{vraag}
        Hedonisme
    \end{vraag}

    \begin{antwoord}
        Hedonimse is de leer binnen de ethiek dat genot het hoogste goed is.
        \begin{citaat}{Wikipedia: ``Hedonism''}
            Hedonism is a school of thought that argues that pleasure is the primary or most important intrinsic good. In very simple terms, a hedonist strives to maximize net pleasure (pleasure minus pain).

            Ethical hedonism is the idea that all people have the right to do everything in their power to achieve the greatest amount of pleasure possible to them, assuming that their actions do not infringe on the equal rights of others.
            It is also the idea that every person's pleasure should far surpass their amount of pain.
            Ethical hedonism is said to have been started by Aristippus of Cyrene, a student of Socrates.
            He held the idea that pleasure is the highest good.
        \end{citaat}
        \begin{citaat}{Borderline times: Over het hedonisme (de genotscultuur) (p. 67)}
            Er lijkt nog slechts één universeel gebod, en dat is: ‘Genieten!’ Zoveel mogelijk.
            Van alles.
            De economisch-consumentalistische realiteit, die - zoals we hierboven bespraken - mee afstraalt op onze relationele werkelijkheid, legt ons ook de druk van het hedonisme op.
            Het leven ‘moet’ leuk zijn, relaties ‘moeten’ leuk zijn... en wanneer het niet leuk is, dan is het in onze opdracht van zelfcreatie en zelfrealisatie noodzakelijk om zelf en individueel een andere beslissing re nemen.
            We ‘moeten voor onszelf zorgen’ en ‘het leven voor onszelf leuk houden’.
            Genieten is in dit verhaal niet enkel de nieuwe moraal geworden, het is ook ieders plicht.
            ‘Gij zult genieten’ heeft nagenoeg alle andere geboden van de troon gestoten.
        \end{citaat}
    \end{antwoord}
\end{examenvraag}


\begin{examenvraag}
    \begin{vraag}
        Ervaringshonger
    \end{vraag}

    \begin{antwoord}
        De mens is constant op zoek naar nieuwe, betere  en meer ervaringsprikkels. Dit in een verlangen naar identiteit, geluk en zin.
        \begin{citaat}{Slides: `Wat kenmerkt onze maatschappij? (34)'}
            Toenemende ervaringshonger:
            \begin{itemize}
                \item = de postmoderne mens wil (en moet!) beleven, ervaren, genieten, plezier maken, voelen dat hij/zij leeft (= de ``genotscultuur'' of ``kickcultuur'')
                \item Kan de postmoderne mens nog verlangen? Of moet het allemaal “nu” en “onmiddellijk”?
                \item De postmoderne mens is altijd op zoek naar het nieuwe, datgene dat hij/zij nog niet meegemaakt heeft (en dat moet steeds straffer en specialer zijn).
                \item De postmoderne mens verdraagt het gewone, het banale, het alledaagse maar moeilijk
            \end{itemize}
        \end{citaat}	
        \begin{citaat}{Onderbroken traditie (p. 71)}
            In zijn of haar ervaringshonger is hij of zij steeds op zoek naar meer, sneller, sterker.
        \end{citaat}
    \end{antwoord}
\end{examenvraag}


\begin{examenvraag}
    \begin{vraag}
        Kick
    \end{vraag}

    \begin{antwoord}
        Een prikkel.
        Een pseudo-ervaring.
        Levert een verheviging van het ik-bewustzijn, een verdichting van het nu-moment.
        Een momentane bevestiging. Zie ook blz 70-71 boek.
        \begin{citaat}{Slides: `Wat kenmerkt onze maatschappij? (33)'}
            De strategie = zich overleveren aan intense ervaringen die een momentane bevestiging verschaffen van het eigen ik.
        \end{citaat}
        \begin{citaat}{Slides: `Wat kenmerkt onze maatschappij? (35)'}
            Een kick stelt het onzekere `ik' in staat zichzelf momentaan te bevestigen: ``ik ben hier'', ``ik best'', ``ik leef echt'', ``ik ga niet op in de vormloze massa'', ``ik ben een individu''
        \end{citaat}
        \begin{citaat}{Onderbroken traditie (p. 70-71)}
            De cultuur van de kick\\
            Onze omwereld - zoals blijkt uit vele life-style-magazines - lijkt te suggereren dat de kick de postmoderne ervaringsmodus geworden is, met het elastiekspringen als voorbeeld bij uitnemendheid.
            `Kick' betekent trap, shop, stoot; hier; sensatie, prikkel, verhevigde schokbeleving, de als aangenaam ervaren schok.
            De postmoderne mens lijkt verslaafd aan kicks, aan prikkels.
            In zijn of haar ervaringshonger is hij of zij steeds op zoek naar meer, sneller, sterker.
            Het is alsof de opeenstapeling van kicks aan het individu de verzekering moet geven dat hij of zij niet zal ondergaan in de stroom van de gebeurtenissen, dat hij of zij niet gewoon `meer van hetzelfde' maar `anders' is.
            De kick levert een verheviging van het ik-bewustzijn, een soort verdichting van het nu-moment.
            Het is voor het onzekere `ik' een momentane bevestiging van het `ik ben hier', `ik ga niet ten onder in de onoverzichtelijke brij van veelheid en ori\"enteringsloosheid'.
        \end{citaat}
    \end{antwoord}
\end{examenvraag}


\begin{examenvraag}
    \begin{vraag}
        Lichaamscultuur
    \end{vraag}
    \begin{antwoord}
        Zie citaat. Nogmaals het bewijs dat onze identiteit van 'verhaal' naar 'beeld' is verschoven.
        \begin{citaat}{Slides: `Wat kenmerkt onze maatschappij? (36)'}
            Het lichaam wordt getraind, versierd, veranderd om erbij te horen \& uit te drukken wie men is/wil zijn (fitness en 
            bodybuilding, tatoeages en piercings, plastische chirurgie en esthetische operaties)
        \end{citaat}	
    \end{antwoord}
\end{examenvraag}


\begin{examenvraag}
    \begin{vraag}
        Hypecultuur
    \end{vraag}

    \begin{antwoord}
        Na een innemende gebeuretenis,‭ ‬waaraan veel belang wordt gehecht in de media,‭ ‬kan deze 
        uitgroeten tot een‭ ‬hype.‭ ‬Mensen zijn ermee bezig,‭ ‬laten hun stem horen,‭ ‬komen op straat,‭…‬ 
        Maar even snel als de hype gekomen is,‭ ‬verdwijnt ze ook weer.‭ ‬Enkele jaren later hecht 
        niemand er nog belang aan,‭ ‬onafhankelijk of de gebeurtenis verdere gevolgen heeft gekend op 
        politiek,‭ ‬sociaal of economisch vlak.‭ ‬Vb.‭ ‬Na de zaak-Dutroux was er een Witte Mars waarin 
        men opperde voor meer aandacht aan het kind.‭ ‬Enkele jaren later bleek er politiek weinig 
        veranderd aan deze kwestie maar niemand maalde erom.‭ ‬De realiteit ging gewoon verder.
        \begin{citaat}{Borderline times (p. 183)}
            De media bepalen de werkelijkheid, de `waarheid' in onze cultuur lijkt meer en meerr een constructie van de media- (en de reclame)wereld.
            Plots ontstaan er hypes.
        \end{citaat}
    \end{antwoord}
\end{examenvraag}


\begin{examenvraag}
    \begin{vraag}
        Mediatisering
    \end{vraag}

    \begin{antwoord}
        De impact van een gebeurtenis wordt afgemeten naargelang de media er aandacht aan hecht.‭ ‬Zo 
        kunnen gebeurtenissen enorm uitvergroot worden maar ook genegeerd.‭ ‬En vanaf het moment dat 
        ze niet meer in‭ “‬Het Nieuws‭”‬ komen,‭ ‬zijn ze verdwenen en gaan we verder met het dagelijkse 
        bestaan.‭ ‬We staan niet stil op wat we er als maatschappij nadien mee zullen aanvangen of 
        wat de invloed er van is op onze toekomst.
        \begin{citaat}{Borderline times (p. 183)}
            De media bepalen de werkelijkheid, de `waarheid' in onze cultuur lijkt meer en meerr een constructie van de media- (en de reclame)wereld.
        \end{citaat}
        \begin{citaat}{Borderline times (p. 184)}
            Niet de omvang van de gebeurtenissen zelf, maar de ruimte die de media eraan geven, bepaalt de grote van de inpact die ze hebben op het bestaan.
        \end{citaat}
    \end{antwoord}
\end{examenvraag}


\begin{examenvraag}
    \begin{vraag}
        Economisering
    \end{vraag}

    \begin{antwoord}
    	Alles staat in functie van de economie.‭ ‬Wat economisch niets oplevert,‭ ‬is niet belangrijk.
        Het vergroten van de rol van markten en contracten als co\"ordinatiemechanismen, zodat een 
        minder groot beroep hoeft te worden gedaan op normen bij het co\"ordineren van 
        activiteiten. Zie ook citaat.
        \begin{citaat}{Slides: ``Wat kenmerkt onze maatschappij (40)''}
            Economisering = in steeds meer domeinen gedragen mensen zich als “consumenten” (die shoppen) en worden ze benaderd als “cliënten” (“klant is koning”) door “leveranciers” die hun product zo goed mogelijk in de markt proberen te plaatsen (door het zo aantrekkelijk mogelijk voor te stellen).
        \end{citaat}
    \end{antwoord}
\end{examenvraag}

\begin{examenvraag}
    \begin{vraag}
        Vermarkting
    \end{vraag}

    \begin{antwoord}
    	Alles is te koop.‭ ‬Mensen gedragen zich als consumenten en worden behandeld als cliënten waaraan leveranciers hun producten proberen te verkopen door ze zo aantrekkelijk mogelijk voor te stellen.
        \begin{citaat}{Slides: ``Wat kenmerkt onze maatschappij (40)''}
            Economisering = in steeds meer domeinen gedragen mensen zich als “consumenten” (die shoppen) en worden ze benaderd als “cliënten” (“klant is koning”) door “leveranciers” die hun product zo goed mogelijk in de markt proberen te plaatsen (door het zo aantrekkelijk mogelijk voor te stellen).
        \end{citaat}
    \end{antwoord}
\end{examenvraag}

\begin{examenvraag}
    \begin{vraag}
        Rank and Yank
    \end{vraag}

    \begin{antwoord}
        Het fenomeen waar een bedrijf (of abstracter, de maatschappij) zijn werknemers onderling een ranking toekent, waarbij de laagsten zullen worden verwijderd.
        Dit verhoogt de concurrentie, cre\"eert wantrouwen onderling en zorgt voor angst en depressie.
        \begin{citaat}{Paul Verhaeghe over neoliberalisme}
            De manier om die zogenaamde effici\"entie te bereiken, is het installeren van concurrentie tussen werknemers, en - ruimer - tussen mensen in het algemeent.
            Dit is het Rank en Yank systeem zoals het ingevoerd werd door het Amerikaanse Enronbedrijf: evalueer alle werknemers op hun productiviteit, vergelijk ze onderling, en ontsla jaarlijks de twintig procent laagst scorenden nadat je ze eerst belachelijk hebt gemaakt.
            [...]
            Het neoliberale systeem gaat op een systematische manier alle sociale verbanden doorknippen.
            Het is het individu dat ge\"evalueerd wordt,  dat  al  dan  niet  een  bonus  krijgt  of  een  individuele  trajectbegeleiding,  enzovoort.
            Het  andere individu is daarbij een potentiële bedreiging en steeds een concurrent binnen een veralgemeende Rank and Yank gemeenschap (Hoeveel likes heb jij op je facebookpagina? En hoeveel bezoekers op je blog?).
            Een typisch gevolg daarvan is de exponenti\"ele toename van contracten, als uitdrukking van ons veralgemeend wantrouwen. Het doorgedreven individualisme levert vandaag heel veel eenzaamheid op, als pijnlijkste symptoom van onze tijd.
        \end{citaat}
    \end{antwoord}
\end{examenvraag}


\begin{examenvraag}
    \begin{vraag}
        Loonspanning
    \end{vraag}

    \begin{antwoord}
        Het onderscheid tussen hoogste en laagste inkomens.
        \begin{citaat}{Paul Verhaeghe over neoliberalisme}
            Een neoliberale maatschappij doet de zogenaamde loonspanning, het onderscheid tussen de hoogste en laagste inkomens, sterk stijgen.
        \end{citaat}
    \end{antwoord}
\end{examenvraag}


\begin{examenvraag}
    \begin{vraag}
        Crowding out
    \end{vraag}

    \begin{antwoord}
	Het verdwijnen van intrinsieke motivaties door het gebruik van extrinsieke incentives zoals 
	geldelijke belongingen vb.‭ ‬je rijdt bij je grootouders enkel het gras af omdat je er geld voor 
	krijgt.‭ ‬Niet omdat je ze wil helpen met dit zware werk.
    \end{antwoord}
\end{examenvraag}


\begin{examenvraag}
    \begin{vraag}
        Moraal
    \end{vraag}

    \begin{antwoord}
    	Het geheel van normen en waarden. 
		\begin{citaat}{Slides Wat is ethiek(2)}
			'het geheel van de (door een groep of cultuur) geaccepteerde gedragsregels', 
			'het geheel van opvattingen, beslissingen en handelingen waarmee mensen uitdrukken wat zij goed of behoorlijk vinden',
			'het geheel van normen en waarden die feitelijk bestaan in een samenleving.
		\end{citaat}
    \end{antwoord}
\end{examenvraag}


\begin{examenvraag}
    \begin{vraag}
        Descriptieve ethiek
    \end{vraag}

    \begin{antwoord}
    	Pure objectieve beschrijving van moraal.
    	Zonder een of ander standpunt of oordeel. 
		\begin{citaat}{Slides Wat is ethiek(3)}
			de beschrijving van de heersende moraal, met de beschrijving van zeden en gewoonten, opvattingen over goed en kwaad,
			verantwoord en onverantwoord gedrag, en geoorloofde en ongeoorloofde handelingen.
		\end{citaat}
    \end{antwoord}
\end{examenvraag}


\begin{examenvraag}
    \begin{vraag}
        Prescriptieve ethiek
    \end{vraag}

    \begin{antwoord}
	    	
    	Oordeelt over de huidige moraal.
    	Gaat na of de huidige moraal wel voldoet aan bepaalde waarden. 
    	'Beantwoorden de huidige gehanteerde normen en waarden wel aan de opvattingen over hoe mensen zich behoren te gedragen?'
    		
		\begin{citaat}{Slides Wat is ethiek(3)}
				De studie die nagaat wat het meest wenselijke handelen is. 
				Daarbij wordt gerefereerd naar ‘het goede’ als criterium.
		\end{citaat}
		\begin{citaat}{Slides Wat is ethiek(3)}
				Prescriptieve ethiek oordeelt over de moraal.
		\end{citaat}
    \end{antwoord}
\end{examenvraag}


\begin{examenvraag}
    \begin{vraag}
        Normatieve ethiek
    \end{vraag}

    \begin{antwoord}
		Zie prescriptieve ethiek.
    \end{antwoord}
\end{examenvraag}


\begin{examenvraag}
    \begin{vraag}
        Empathie
    \end{vraag}

    \begin{antwoord}
		Synoniem voor inlevingsvermogen.
		In de cursus wordt het begrip in 2 delen ontleed.
		\begin{itemize}
			\item Cognitief
				Het kunnen begrijpen en delen van andersmans gevoelens.
			\item Emotioneel
				Het kunnen overnemen(spiegelen) van andersmans gevoelens. (geeuwen, wenen bij babies)
		\end{itemize}
    \end{antwoord}
\end{examenvraag}


\begin{examenvraag}
    \begin{vraag}
        De zuilen van de moraliteit
    \end{vraag}

    \begin{antwoord}
		Moraliteit is erop gebaseerd.
		Een is wederkerigheid, met daaraan gekoppeld een notie van rechtvaardigheid. 
		De andere is empathie en compassie. 
		Menselijke moraliteit is meer dan dit, maar deze 2 zuilen zijn absoluut essentieel.
    \end{antwoord}
\end{examenvraag}


\begin{examenvraag}
    \begin{vraag}
        De Baal Sjem Tov
    \end{vraag}

    \begin{antwoord}
        Voor de joden is het belangrijker te doen dan te weten.‭ ‬Baal Sjem Tov was een Joods rabbijn 
        dat‭ ‬onbewust en ongewild een overtreding had begaan.‭ ‬God liet hem in een droom weten dat‭ 
        ‬hem de toegang tot het paradijs werd ontzegd.‭ ‬Dit was voor Baal Sjem Tov een enorme 
        opluchting.‭ ‬Nu zijn leven geen zin meer had zou hij de Heer kunnen dienen omwille van zijn 
        Heer-zijn en niet langer omwille van het hiernamaals of een beloning.‭ ‬Het is dus‭ ‬niet‭ “‬de 
        zin‭”‬ dat centraal staat maar het onvoorwaardelijk houden van God en handelen volgens het 
        goede.
    \end{antwoord}
\end{examenvraag}


\begin{examenvraag}
    \begin{vraag}
        Dubbelgebod van de liefde
    \end{vraag}

    \begin{antwoord}
        Gij zult de Heer uw God beminnen met geheel uw hart en geheel uw ziel, 
        met al uw krachten en geheel uw verstand; 
        en uw naaste gelijk uzelf.
    \end{antwoord}
\end{examenvraag}


\begin{examenvraag}
    \begin{vraag}
        Verantwoordelijkheid in de eerste persoon
    \end{vraag}

    \begin{antwoord}
        Verantwoordelijkheidsgevoel voor je eigen ontwikkeling.
        Zorgen dat je je handhaaft in het bestaan.
        Een beetje egocentrisme, om je eigen doelen(verhaal) na te streven.
    \end{antwoord}
\end{examenvraag}


\begin{examenvraag}
    \begin{vraag}
        Verantwoordelijkheid in de tweede persoon
    \end{vraag}

    \begin{antwoord}
        Verantwoordelijkheidsgevoel voor een ander (in nood).
        Begaan zijn met de emotionele/fysieke welgesteldheid van een medemens.
        Letterlijk: mijn raakbaarheid voor de lijdende ander.
        Als je deze verantwoordelijkheid opneemt spreekt men van barmhartigheid.

    \end{antwoord}
\end{examenvraag}


\begin{examenvraag}
    \begin{vraag}
        Heteronomie
    \end{vraag}

    \begin{antwoord}
        Synoniem voor 'onderwerping aan vreemde wetten'.
        De verschijning van de ander heb je niet in de hand. (ref barmhartige samaritaan)
        Het is er opeens. 
        Het verstoort je eigen bestaan, je eigen levensweg.
    \end{antwoord}
\end{examenvraag}


\begin{examenvraag}
    \begin{vraag}
        Barmhartigheid
    \end{vraag}

    \begin{antwoord}
	Positief opgenomen verantwoordelijkheid in de‭ ‬2e persoon.‭ ‬Mijn gevoeligheid voor het lijden van 
	de ander effectief voltrekken.‭ ‬Hierdoor ontstaat in mij een beweging naar de ander toe om hem 
	onvoorwaardelijk bij te staan.‭ ‬We kunnen dit eigenlijk ook ethisch moederschap noemen‭ ‬‘in zich 
	dragen van de ander tot hij geboren wordt‭’‬.‭ ‬Barmhartigheid gaat niet uit van een uitwendig 
	verbod maar vanuit een inwendig moeten.
    \end{antwoord}
\end{examenvraag}


\begin{examenvraag}
    \begin{vraag}
        Verantwoordelijkheid in de derde persoon
    \end{vraag}

    \begin{antwoord}
        Verantwoordelijkheidsgevoel voor de derde. 
        De persoon die niet fysiek zichtbaar is.
        Voor de volgende generaties.
    \end{antwoord}
\end{examenvraag}


\begin{examenvraag}
    \begin{vraag}
        Alteriteitservaring
    \end{vraag}

    \begin{antwoord}
	Een soort grenservaring die geen onmiddellijke identiteitservaring is zoals een kick maar 
	eentje die ons overvalt.‭ ‬Het is een ervaring waarover we zelf geen heer en meester zijn.‭ ‬Het 
	zijn radicale gebeurtenissen die onze bestaande verhalen,‭ ‬opgebouwd op evidenties door elkaar 
	schudden en kunnen zowel positief als negatief zijn.‭ ‬Als grens aan onze eigen identiteit 
	nodigen alteriteitservaringen ons uit onszelf te laten raken.‭ ‬Dit betekent dat we niet meteen 
	mogen proberen de vreemdheid onze eigen wil te maken.‭ ‬Uiteindelijk vinden we hierin onze 
	identiteit in het steeds weggeroepen worden uit de beslotenheid van ons eigen verhaal.
    \end{antwoord}
\end{examenvraag}


\begin{examenvraag}
    \begin{vraag}
        De “list” waarvan sprake in Schindler’s List
    \end{vraag}

    \begin{antwoord}
        Oskar Shindler heeft een duizendtal gevangenen, vooral joden, geredt uit het concentratiekamp van Plaseow door ze te werk te stellen in zijn fabriek in Krakow.
        Hij kocht de joden van de kampcommandant.
        Hij bezorgde de commandant een lijst met fictieve jobs voor de joden die hij kocht om de commandant te overtuigen dat de joden van vitaal belang waren, als werkkracht, voor de oorlog.
        Hij redde zo 1200 joden van de dood in het concentratiekamp.
        "The list is absolute good. The list is life, around the edges is a gulf", wat zoveel betekent als "Diegenen die niet op de lijst stonden overleefden niet.", is hieromtrent een markant citaat.
    \end{antwoord}
\end{examenvraag}


\begin{examenvraag}
    \begin{vraag}
        Kloofmodel
    \end{vraag}

    \begin{antwoord}
	Het model van oa.‭ ‬Herman De Dijn‭ ‬waarbij hij een strikte kloof stelt tussen levensbeschouwing 
	en wetenschap.‭ ‬Het is het onderscheid tussen cognitieve en zingevende interesse.‭ ‬Hierbij staat 
	cognitieve interesse voor manipulatieve interesse‭; ‬om de werkelijkheid te veranderen moet ze 
	eerst‭ ‬gekend worden.‭ ‬De zingevende interesse gaat om het in-waarheid-leven.‭ ‬Volgens De Dijn 
	kunnen zingeving en wetenschap niet met elkaar in conflict staat omdat ze hiervoor te 
	verschillend zijn.‭ ‬Ze zijn dus niet revaliserend maar bieden twee verschillende manieren om 
	naar de werkelijkheid te kijken met elk een eigen taalregister.
    \end{antwoord}
\end{examenvraag}


\begin{examenvraag}
    \begin{vraag}
        Het NOMA-principe
    \end{vraag}

    \begin{antwoord}
	Non-Overlapping Magisteria.‭ ‬Geloof en wetenschap zijn twee gescheiden,‭ ‬evenwaardige en 
	complementaire ondernemingen met een strikte taakverdeling.‭ ‬Er is geen conflict tussen beide 
	zolang ze zich ieder aan hun vakgebied houden.‭ ‬Zo mag de wetenschap geen uitspraken doen over 
	ethische of religieuze thema‭’‬s en mogen religieuze inzichten niet in tegenspraak zijn met 
	wetenschappelijke feiten.
    \end{antwoord}
\end{examenvraag}


\begin{examenvraag}
    \begin{vraag}
        Stephen Jay Gould
    \end{vraag}

    \begin{antwoord}
	Een natuurwetenschapper die een onderscheid maakt tussen natuurwetenschappen en geloof in die 
	zin dat het twee gescheiden,‭ ‬evenwaardige en complementaire ondernemingen zijn.‭ ‬Om deze niet 
	met elkaar te verwarren,‭ ‬moeten ze zich elk aan een strikte taakverdeling houden waardoor ze 
	niet met elkaar in conflict staan‭ = ‬het NOMA-principe.
    \end{antwoord}
\end{examenvraag}


\begin{examenvraag}
    \begin{vraag}
        Harmoniemodel
    \end{vraag}

    \begin{antwoord}
	Het model waarbij‭ ‬gesteld wordt dat de materialistische interpretatie van de evolutietheorie 
	niet dwingend is maar evenzeer geïntegreerd kan worden in een theologisch‭ ‬wereldbeeld.‭ ‬Het feit 
	dat God een systematische afslachting van de zwakkeren toestaat,‭ ‬bewijst dat hij geen 
	almachtige heerser is maar een kwetsbare partner die ons mogelijkheden biedt en het universum 
	de kans geeft zichzelf te realiseren.
    \end{antwoord}
\end{examenvraag}


\begin{examenvraag}
    \begin{vraag}
        Differentiemodel
    \end{vraag}

    \begin{antwoord}
	Het model waarbij er geen kloof bestaat tussen geloof en wetenschap maar een constitutief 
	verschil.‭ ‬Er is een open dialoog waarbij de gelovige de wetenschap ziet als een uitdaging voor 
	het eigen verhaal.‭ ‬Hierin is de wetenschap een partner in de zoektocht naar het beter begrijpen 
	van de mens en wereld.‭ ‬Geloof en wetenschap hebben een eigen taal en taalregister.‭ ‬Deze 
	taalregisters hebben elk hun eigen plaats en kunnen niet van plaats verwisselen.‭ 
    \end{antwoord}
\end{examenvraag}



\end{document}
