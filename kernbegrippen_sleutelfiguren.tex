\documentclass[main.tex]{subfiles}

\begin{document}
\section{Kernbegrippen en sleutelfiguren}


\begin{examenvraag}
    \begin{vraag}
        Identificatie
    \end{vraag}

    \begin{antwoord}
        TODO
        Identificatie, spiegeling, tegenover separatie.
    \end{antwoord}
\end{examenvraag}


\begin{examenvraag}
    \begin{vraag}
        TODO
        Separatie
    \end{vraag}

    \begin{antwoord}

    \end{antwoord}
\end{examenvraag}


\begin{examenvraag}
    \begin{vraag}
        Eudaimonia
    \end{vraag}

    \begin{antwoord}

    \end{antwoord}
\end{examenvraag}


\begin{examenvraag}
    \begin{vraag}
        Detraditionalisering
    \end{vraag}

    \begin{antwoord}

    \end{antwoord}
\end{examenvraag}


\begin{examenvraag}
    \begin{vraag}
        Keuzebiografie
    \end{vraag}

    \begin{antwoord}

    \end{antwoord}
\end{examenvraag}


\begin{examenvraag}
    \begin{vraag}
        Radicale pluraliteit
    \end{vraag}

    \begin{antwoord}

    \end{antwoord}
\end{examenvraag}


\begin{examenvraag}
    \begin{vraag}
        Het ideaal van de romantische liefde
    \end{vraag}

    \begin{antwoord}

    \end{antwoord}
\end{examenvraag}


\begin{examenvraag}
    \begin{vraag}
        Wegwerpcultuur
    \end{vraag}

    \begin{antwoord}

    \end{antwoord}
\end{examenvraag}


\begin{examenvraag}
    \begin{vraag}
        Hedonisme
    \end{vraag}

    \begin{antwoord}

    \end{antwoord}
\end{examenvraag}


\begin{examenvraag}
    \begin{vraag}
        Ervaringshonger
    \end{vraag}

    \begin{antwoord}

    \end{antwoord}
\end{examenvraag}


\begin{examenvraag}
    \begin{vraag}
        Kick
    \end{vraag}

    \begin{antwoord}

    \end{antwoord}
\end{examenvraag}


\begin{examenvraag}
    \begin{vraag}
        Lichaamscultuur
    \end{vraag}

    \begin{antwoord}

    \end{antwoord}
\end{examenvraag}


\begin{examenvraag}
    \begin{vraag}
        Hypecultuur
    \end{vraag}

    \begin{antwoord}

    \end{antwoord}
\end{examenvraag}


\begin{examenvraag}
    \begin{vraag}
        Mediatisering
    \end{vraag}

    \begin{antwoord}

    \end{antwoord}
\end{examenvraag}


\begin{examenvraag}
    \begin{vraag}
        Economisering
    \end{vraag}

    \begin{antwoord}

    \end{antwoord}
\end{examenvraag}


\begin{examenvraag}
    \begin{vraag}
        Vermarkting
    \end{vraag}

    \begin{antwoord}

    \end{antwoord}
\end{examenvraag}


\begin{examenvraag}
    \begin{vraag}
        Rank and Yank
    \end{vraag}

    \begin{antwoord}

    \end{antwoord}
\end{examenvraag}


\begin{examenvraag}
    \begin{vraag}
        Loonspanning
    \end{vraag}

    \begin{antwoord}

    \end{antwoord}
\end{examenvraag}


\begin{examenvraag}
    \begin{vraag}
        Crowding out
    \end{vraag}

    \begin{antwoord}

    \end{antwoord}
\end{examenvraag}


\begin{examenvraag}
    \begin{vraag}
        Moraal
    \end{vraag}

    \begin{antwoord}

    \end{antwoord}
\end{examenvraag}


\begin{examenvraag}
    \begin{vraag}
        Descriptieve ethiek
    \end{vraag}

    \begin{antwoord}

    \end{antwoord}
\end{examenvraag}


\begin{examenvraag}
    \begin{vraag}
        Prescriptieve ethiek
    \end{vraag}

    \begin{antwoord}

    \end{antwoord}
\end{examenvraag}


\begin{examenvraag}
    \begin{vraag}
        Normatieve ethiek
    \end{vraag}

    \begin{antwoord}

    \end{antwoord}
\end{examenvraag}


\begin{examenvraag}
    \begin{vraag}
        Empathie
    \end{vraag}

    \begin{antwoord}

    \end{antwoord}
\end{examenvraag}


\begin{examenvraag}
    \begin{vraag}
        De zuilen van de moraliteit
    \end{vraag}

    \begin{antwoord}

    \end{antwoord}
\end{examenvraag}


\begin{examenvraag}
    \begin{vraag}
        De Baal Sjem Tov
    \end{vraag}

    \begin{antwoord}

    \end{antwoord}
\end{examenvraag}


\begin{examenvraag}
    \begin{vraag}
        Dubbelgebod van de liefde
    \end{vraag}

    \begin{antwoord}

    \end{antwoord}
\end{examenvraag}


\begin{examenvraag}
    \begin{vraag}
        Verantwoordelijkheid in de eerste persoon
    \end{vraag}

    \begin{antwoord}

    \end{antwoord}
\end{examenvraag}


\begin{examenvraag}
    \begin{vraag}
        Verantwoordelijkheid in de tweede persoon
    \end{vraag}

    \begin{antwoord}

    \end{antwoord}
\end{examenvraag}


\begin{examenvraag}
    \begin{vraag}
        Heteronomie
    \end{vraag}

    \begin{antwoord}

    \end{antwoord}
\end{examenvraag}


\begin{examenvraag}
    \begin{vraag}
        Barmhartigheid
    \end{vraag}

    \begin{antwoord}

    \end{antwoord}
\end{examenvraag}


\begin{examenvraag}
    \begin{vraag}
        Verantwoordelijkheid in de derde persoon
    \end{vraag}

    \begin{antwoord}

    \end{antwoord}
\end{examenvraag}


\begin{examenvraag}
    \begin{vraag}
        Alteriteitservaring
    \end{vraag}

    \begin{antwoord}

    \end{antwoord}
\end{examenvraag}


\begin{examenvraag}
    \begin{vraag}
        De “list” waarvan sprake in Schindler’s List
    \end{vraag}

    \begin{antwoord}
        Oskar Shindler heeft een duizendtal gevangenen, vooral joden, geredt uit het concentratiekamp van Plaseow door ze te werk te stellen in zijn fabriek in Krakow.
        Hij kocht de joden van de kampcommandant.
        Hij bezorgde de commandant een lijst met fictieve jobs voor de joden die hij kocht om de commandant te overtuigen dat de joden van vitaal belang waren, als werkkracht, voor de oorlog.
        Hij redde zo 1200 joden van de dood in het concentratiekamp.
        "The list is absolute good. The list is life, around the edges is a gulf", wat zoveel betekent als "Diegenen die niet op de lijst stonden overleefden niet.", is hieromtrent een markant citaat.
    \end{antwoord}
\end{examenvraag}


\begin{examenvraag}
    \begin{vraag}
        Kloofmodel
    \end{vraag}

    \begin{antwoord}

    \end{antwoord}
\end{examenvraag}


\begin{examenvraag}
    \begin{vraag}
        Het NOMA-principe
    \end{vraag}

    \begin{antwoord}

    \end{antwoord}
\end{examenvraag}


\begin{examenvraag}
    \begin{vraag}
        Stephen Jay Gould
    \end{vraag}

    \begin{antwoord}

    \end{antwoord}
\end{examenvraag}


\begin{examenvraag}
    \begin{vraag}
        Harmoniemodel
    \end{vraag}

    \begin{antwoord}

    \end{antwoord}
\end{examenvraag}


\begin{examenvraag}
    \begin{vraag}
        Differentiemodel
    \end{vraag}

    \begin{antwoord}

    \end{antwoord}
\end{examenvraag}



\end{document}
