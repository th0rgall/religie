\documentclass[main.tex]{subfiles}

\begin{document}
\section{Kernbegrippen en sleutelfiguren}

\subsection*{Identificatie}
\subsection*{Separatie}
\subsection*{Eudaimonia}
\subsection*{Detraditionalisering}
\subsection*{Keuzebiografie}
\subsection*{Radicale pluraliteit}
\subsection*{Het ideaal van de romantische liefde}
\subsection*{Wegwerpcultuur}
\subsection*{Hedonisme}
\subsection*{Ervaringshonger}
\subsection*{Kick}
\subsection*{Lichaamscultuur}
\subsection*{Hypecultuur}
\subsection*{Mediatisering}
\subsection*{Economisering}
\subsection*{Vermarkting}
\subsection*{Rank and Yank}
\subsection*{Loonspanning}
\subsection*{Crowding out}
\subsection*{Moraal}
\subsection*{Descriptieve ethiek}
\subsection*{Prescriptieve ethiek}
\subsection*{Normatieve ethiek}
\subsection*{Empathie}
\subsection*{De zuilen van de moraliteit}
\subsection*{De Baal Sjem Tov}
\subsection*{Dubbelgebod van de liefde}
\subsection*{Verantwoordelijkheid in de eerste}
\subsection*{persoon}
\subsection*{Verantwoordelijkheid in de tweede}
\subsection*{persoon}
\subsection*{Heteronomie}
\subsection*{Barmhartigheid}
\subsection*{Verantwoordelijkheid in de derde}
\subsection*{persoon}
\subsection*{Alteriteitservaring}
\subsection*{De “list” waarvan sprake in Schindler’s List}
\subsection*{Kloofmodel}
\subsection*{Het NOMA-principe}
\subsection*{Stephen Jay Gould}
\subsection*{Harmoniemodel}
\subsection*{Differentiemodel}
\end{document}